\chapter{Konklusion}
Sikkerheden i, at ens live har retning mod døden påvirke utvivlsomt vores tanker om livet, og om der er mere til livet, end det vi erfarer her på jorden. Om det sidstnævnte kan man sige, at det opfordre til spørgsmålet: Hvad kommer der til at ske med os efter døden. Fortstående opgave redegør for fem forestillinger, der varierer fra hinanden i forhold til, hvordan Gud i sin nåde forestilles at møde mennesket. Forestillingerne har hver især deres problem områder og deres lyspunkter. Når alt kommer til alt er menneskets fortolkning af verdenen blot det, en fortolkning. Det vil sige, at der ingen konklusion er på spørgsmålet, om vi kommer at opleve en skæbne i evigheden eller om døden blot er den blotte død. Overfor et sådan dilemma forklarer udsagnet "netop over for evigheden bliver tiden dyrebar"\footnote{Gregersen, 2000, s. 134}, hvordan døden giver menneskets livsøjeblikke en dybere værdi og personlig betydning. Mennesket er i denne forstand netop nødt til at forholde sig til det liv, som er nu. Fordi det ved ikke, hvad evigheden har med sig af fylde eller ingenting. Den guddommelige stilhed, som nævnt i tesen, kommer i den forstand at være et udtryk for accept af menneskets ret til dets egne forståelse af livet og det, der følger efter døden. I denne sammenhæng giver det mening med Talbotts udsagn om, at når alt kommer til alt, er det op til Gud, hvordan han på bedste vis åbenbarer sig for os. 

Man kan sige, at de menneskelige erfaringer giver udtryk for, at livet netop er et eksperiment, og det er et dødsseriøst eksperiment. I dette tilfælde er frelsesuniversalismen fortrøstningsfuld -- i den kan vi finde håbet om en frelse, der tillader os at "live wholly here, and die wholly, and rise wholly there."\footnote{Moltmann, 1996, s. 67}