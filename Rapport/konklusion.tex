\chapter{Konklusion}
Konklusionen er, at der ingen konklusion er for, hvilken domsforestilling kommer tættest på sandheden. I indledningen til sin artikel skriver Gregersen, at ”netop over for evigheden bliver tiden dyrebar.”1 Er det ikke netop det, der er den højeste værdi, at forestillingerne blot er forestillinger? Mennesket er derved frit stillet til at fortolke livet og døden. Jeg forstår den guddommelig stilhed som en respekt menneskets ret til egen forståelse af livets mening. Cliff-hanger er i den forstand en forstyrrelse, men her kommer forestillingerne ind i billedet som et tilbud om mulig fortolkning af meningen med livet. I CG giver Moltmann udtryk for at ”life is not an experiment.” Dette synspunkt er set ud fra håbet om alles frelse. Men ud fra mine menneskelige erfaringer ville jeg vove at sige, at livet netop er et eksperiment – et dødsseriøst eksperiment. Håbet er, at eksperimentet giver mening nok til, at man, når dødens er nær, kan finde ro med at passere denne definitive dørtærskel. I tesen satte jeg spørgsmål med meningen i forestillingerne. Konklusion er, at det for nogle mennesker giver en sans for dybere mening i livet at håbe på noget. Teologen Thomas Talbott udtrykt det således, at når alt kommer til alt, er det op til Gud, hvordan han på bedste vis vil åbenbarer sig for os.2