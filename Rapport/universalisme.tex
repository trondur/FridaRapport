\chapter{Universalisme}
I bøgerne The Coming of God (CG) og Theology of Hope (TH) redegør Moltmann for sin opfattelse af den kristne eskatologi. Indenfor den teologiske retning kaldet apokatastasis repræsenterer han tanken om universalisme, som omfatter tanken om håbet på en universel frelse og restaurationen af alle ting. Spørgsmålet om alles frelse er et eskatologisk spørgsmål, som, ifølge Moltmann, teologisk set kun kan besvares kristologisk.\footnote{Moltmann, 1996, s. 237 og 250} Dette bliver også klart igennem hans eskatologiske tænkning, hvor Kristus på korset opfattes som svaret på det eskatologiske håb.

Fænomenet håb står centralt i Moltmanns forestilling om universalisme. Under betegnelsen 'personlig eskatologi' redegør han for individets personlige håb på evigt liv. Historisk eskatologi spiller derimod en rolle for individet i forhold til det universelle håb. Dette ud fra opfattelsen af, at mennesket er en integreret del af en større sammenhæng.\footnote{Moltmann, 1996, s. 131} Men ifølge Moltmann hænger den personlige og den historiske eskatologi tæt sammen, for det universelle håb skal omfatte håbet om evigt liv for hele skabelsens univers.  

\section{Personlig eskatologi}
Ifølge Moltmann omhandler personlig eskatologi det enkeltes menneskes tro og håb på det evige liv. Det
vil sige, at omdrejningspunktet for håbet er den eskatologiske begivenhed, hvori menneskets blik er rettet mod tidspunktet for Kristi genkomst. Dette er ifølge Moltmann indledningen på fornyelsen af alle ting. Videre synes han, at al teologi skal ses i et eskatologisk perspektiv. For alle de eskatologiske begivenheder -- den universelle herlighed, den sidste dom, opstandelsen af de døde og nyskabelsen af alle ting -- har en lærende funktion for livet.\footnote{Moltmann, 1967, s. 15} Uden eskatologien som del af nutiden har kristendommens lære ingen frugtbar eksistens, idet den berøves den logiske nødvendighed af korset og opstandelsen, Kristi ophøjelse og suverænitet.\footnote{Moltmann, 1967, s.15 } Sammenhængen i dette er, at eskatologiens håb skal være konstituerende i livet her, hvis håbet, som mennesket holder sig til, skal have en reel betydning, når mennesket står ansigt til ansigt med døden. Dette er, hvad Moltmann forstår ved Wolf Biermanns digt ”we have to get life into life.”\footnote{Moltmann, 1996, s. 52}

Moltmann indleder afsnittet \textit{Personlig Eskatologi} (CG) med menneskets ambivalente forhold til liv og død, ved at fremstille Epikurs opfattelse af forholdet mellem liv og død. Ifølge Epikur kan liv og død kun eksistere adskilt fra hinanden. Hvorimod Moltmann gør opmærksom på, at de to fænomener er tæt forbundet i et gensidigt afhængighedsforhold og eksisterer altid på samme tid. Moltmann opfatter livet og døden som fundamentale oplevelser og ikke blot biologiske fænomener,\footnote{Moltmann, 1996, s. 54} som indadtil forudsætter, hvordan mennesket opfatter og reflekterer over livet som helhed. Derfor må en eskatologi, der vedrører det personlige, inddrage hele menneskets levegrundlag for at bevare dens autenticitet i forhold til håbet. Ifølge Moltmann er håbet forudsætningen for, at mennesket på ingen måde kan affinde sig døden som den blotte død. I kraft af håbet forholder mennesket sig utrøsteligt i forventningen af et liv, der overvinder døden,\footnote{Moltmann, 1996, s. 93} og kristendommen må gå denne forventning i møde. Ifølge Moltmann handler kristendom helt og holdent om eskatologi på grund af dets håbs fremadrettede retning mod en forsoning og forløsning. Herfor er Gud omdrejningspunktet og igennem Kristus har Gud åbenbaret sin fremtid, som er en forsoning uden grænser.\footnote{Moltmann, 1996, s. 104} Ifølge Moltmann er oplevelsen af Kristi død og opstandelse formgivende for den kristne tro. Begivenheden på korset og opstandelsen kommer til at fungere som indgangen til det evige liv.\footnote{Moltmann, 1996, s. 69} Ydermere kommer samme begivenhed at stå som grundlag for menneskets håb på en forsoning uden grænser.\footnote{Moltmann, 1996, s. 250}

Det kristne håb er ifølge Moltmann universalistisk i sin karakter, fordi det håber på et liv, der overvinder døden. Det vil sige, at kristendommen håber på en livgivende Gud, der overvinder døden sådan som 1. Kor. 15:54 giver udtryk for.\footnote{Moltmann, 1996, s. 65}  Transformationen har i denne forbindelse betydning for tanken om dødens opstandelse. Ud fra 1. Kor. 15:52 og Phil. 3:21 forstår Moltmann transformation som en helbredelse, forsoning og fuldbyrdelse. Transformationen er mulig i den forstand, at Guds ånd indenfor kristendommen tænkes at være livgivende og dertil den virkende kraft i opstandelsen.\footnote{Moltmann, 1996, s. 71} Ånden muliggør derved en konfiguration af livet som helhed, der medfører, at mennesket i kraft af Kristi opstandelse kan opleve det fysiske liv som både dødeligt og udødeligt på samme tid.\footnote{Moltmann, 1996, s. 71} Det vil sige, at opstandelsen betyder en indoptagelse af det fysiske liv ind i det evige liv, som for mennesket betyder, at fremtiden er medtænkt i nutiden. Denne fremstilling af tiderne er fra Moltmanns side et forsøg på at sætte tiderne i relation til hinanden i forhold til, hvordan mennesket kan tænke kontinuitet og evigt liv ind i diskontinuiteten, som døden i nuværende liv forudsætter.

Moltmann forstår transformation således, at Gud præserverer menneskets selv og transformerer det for at perfektionerer det. Dette på den måde, at mødet med døden fungerer som en tillære eller en ny orientering, hvor mennesket forstår noget nyt i forhold til sig selv og i forhold til Gud.\footnote{Moltmann, 1996, s. 84-85} Ifølge Moltmann er livshistorien og –erindringen i den forstand vigtige at få med ind i det ny transformerede liv. Ligesom Kristus igennem transformationen beholdt sine identifikationsmærkater, således vil mennesket også igennem transformationen bevare sine kendetegn.\footnote{Moltmann, 1996, s. 84} Moltmanns pointe er, at mennesket igennem døden skal opnå sjælero ved det levede liv. Dette på den måde, at mennesket, i mødet med Gud, møder sig selv. Døden er i den forstand ikke slutningen, men en proces. Dette må forstås i lyset af Moltmanns opfattelse af, at alle mennesket har levet halve og ufuldendte liv, og at Gud igennem apokatastasis giver mennesket muligheden for at udleve dit fulde potentiale.\footnote{Moltmann, 1996, s. 117} 

\section{Historisk Eskatologi} 
Historisk eskatologi omfatter et universelt håb om forløsning for ethvert individ samt for hele universet. I denne forbindelse gør Moltmann opmærksom på, at historisk eskatologi afhænger af den kosmiske eskatologi, nyskabelsen af verden. Ydermere er der ifølge Moltmann ingen personlig eskatologi uden en transformation af de kosmiske vilkår. Det vil sige, at transformationen eller forløsningen ikke kun er indsnævret til at omfatte en lille grubbe udvalgte, men at forløsningen omfatter alt liv i universet som helhed.\footnote{Moltmann, 1996, s. 132} Dommedag opfattes af Moltmann som den universelle åbenbaringen af Kristus og fuldbyrdelsen af hans forløsende handling på korset. Ud fra dette gør Moltmann opmærksom på, at det sande grundlag for kristendommens håb på universel frelse er en teologi, hvis omdrejningspunkt er korset. For at forklare fokusset på korset beskriver Moltmann blandt andet Luthers og Hans Urs von Balthasars forestillinger om betydningen af Kristi nedfart til dødsriget. Luther og von Balthasar opfatter Kristi nedfart til helvede som en eksistentiel erfaring. Ifølge Luther omfatter erfaringen, at Guds vrede og forbandelse er over synd og gudløs væren, således at Kristus led helvede på korset for at forsone denne verden med Gud.\footnote{Moltmann, 1996, s. 252} Ifølge Luther blev Kristus synd for at mennesket skulle opnå forsoning. Von Balthasars opfatter Kristi nedfart påskelørdag som alles frelse. I nedfarten oplevede Kristus den absolutte gudsforladthed, således at menneskeheden kunne være omfattet af solidariteten. Det vil sige, at selv i helvedes dybder finder mennesket Kristus som sin frelser og derigennem også Gud.\footnote{Moltmann, 1996, s. 253} Moltmanns pointe i dette er, at der for mennesket er håb at finde i en teologi med fokus på korset. Budkabet i en sådan en teologi er, at ”Christ gave him self up for lost in order to seek all who are lost, and bring them home.”\footnote{Moltmann, 1996, s. 253} En teologi med dette fokus kan i Moltmanns optik kun resultere i restaurationen af alle ting. For det, som Kristus opnår i lidelse, død og opstandelse, vil manifestere sig i herlighed igennem hans parousia.\footnote{Moltmann, 1996, s. 254} Ud fra dette konkluderer Moltmann, at dommedag ikke har med frygt og bæven at gøre, men dommedag skal hellere opfattes som en stor og glædelig sejer. Moltmann gør opmærksom på, at der er to sider af den eskatologiske doktrin om alle tings restauration. For det første omfatter den Guds dom, at alle ting retfærdiggøres, og for det andet så vækkes Guds rige til nyt liv i den restaurerede skabelse.\footnote{Moltmann, 1996, s. 255}