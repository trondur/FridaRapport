\chapter{Læren om dobbelt udgang}
Forestillingen om dobbelt udgang omfatter, at der for mennesket ved dommedag foreligger to udgange med hver sin specifikke destination. Fortalere for denne opfattelse er Regin Prenter og Augustin. Ifølge dem afhænger destinationen af en dom, der er afgørende for hvilken udgang, det enkelte menneske idømmes. Den ene udgang fører til himmelsk frelse, mens den anden fører til evig fortabelse. Fælles for Prenters og Augustins opfattelse er, at Gud har prædestineret udgangen som dobbelt. Hvor de forekommer at være uenige er derimod opfattelsen af Guds nåde i forhold til Guds udvalgte.

\section{Augustin}
I Augustins Ench. og De civ. Dei. 19 – 22 fremgår forestillingen om menneskehedens absolutte skæbne som opdelt i to grupper, de frelste og de fortabte. Baggrunden for denne opfattelse er blandt andet at finde i Augustins refleksion over det ondes eksistens. Dette spørgsmål optager Augustin i stor grad, og præger hans opfattelse af Guds nåde, frelse og dom.

Augustins eskatologiske tanker er præget af skabelsesteologiske motiver. Ifølge ham er mennesket i besiddelse af en naturlig gudserkendelse. Det vil sige, at mennesket, igennem det skabte, kan erkende sin skaber, Gud. Dette synspunkt fremgår i De civ. Dei 22,24. Da syndefaldet indtraf skabte dette splid i naturen, som medførte en uoverensstemmelse imellem den himmelske og jordiske stad. Dette splid forårsagte, at mennesket ikke kunne erkende Gud, uden at Gud i sin nåde hjalp mennesket med at erkende Gud. Dette er ifølge Augustin årsagen til det ondes mulighed for at eksistere. 

Ud fra Augustins opfattelse af Gud forvalter Gud alt, som han selv har skabt. Ud fra dette perspektiv sætter Augustins spørgsmålstegn ved synden, som ifølge hans forestilling er af det onde. Ifølge Augustins anskuelse er Guds vilje uovertruffen, og kan derfor ikke være ond.  Hvordan kan det onde så eksistere? Augustins konklusion er, at det onde ikke kommer fra Gud, men at det eksisterer, fordi Gud tillader det onde at eksistere.\footnote{Augustin, 2002, s. 910 (21,7)} I Ench. forklarer Augustin således, at når Gud tillader det onde at ske, er det af retfærdighed. Derfor kommer det onde i Augustins optik til at være velanbragt på sit sted som forudsætningen for, at Gud i hans almægtighed kan gøre godt med det onde.\footnote{Augustin, 2005, s. 65 (3.11) } På grundlag af denne opfattelse konkluderer Augustin, at Gud tillod det onde at vinde indpas gennem syndefaldet. Følgen heraf var, at menneskets vilje blev beskadiget, og at mennesket ikke længere ville det gode, uden at Gud i sin barmhjertighed og nåde hjalp mennesket til at ville det gode. Da det efter Augustins forestilling er Gud, der bevirker vores vilje, stræben og tro, så bliver ingen frelst, uden at Gud vil frelse dem. Dette, at Gud bevirker aktiviteten i mennesket, er baggrunden for Augustins forståelse af Guds nåde som ufortjent. Som Prenter har formuleret det, så er Augustins nådeslære ufortjent, men ikke ufortjenstfuld. Mennesket er derfor prisgivet Gud, for ifølge Augustin afhænger ens frelse af, at Gud vil ens frelse.

\subsubsection{De civ. Dei 20}
De civ. Dei 20 omfatter Augustins forestilling om den yderste dag. Ifølge ham er den yderste dag regnskabets dag, hvor der opgøreles, hvem har fortjent lykke eller ulykke. Afgørende for dommens dag er Kristi genkomst, da det er ham der skal dømme levende og døde.\footnote{Augustin, 2002, s. 846 (20,1)} Dommens dag åbenbarer, hvem de gode og slette er, fordi de vil hver især blive dømt i forhold til det de fortjener.\footnote{Augustin, 2002, s. 946 (20,1)} Netop dette 'hvad hver enkelt fortjener' er centralt i forhold til Augustins tanke om nåden. Nåden i denne sammenhæng kommer at fremstå som et retsmæssig forhold mellem Gud og mennesket. Hovedfokus i dette forhold er straf og belønning og kommer i store træk til at afgøre, hvem Guds frelse omfatter. Ifølge Augustin er Guds frelsesvilje begrænset til kun at omfatte nogle få udvalgte. Augustin lægger vægt på ret livsførelse som forudsætning for opnåelsen af frelse og det evige liv.\footnote{Augustin, 2002, s. 811 (19,4)} Ret livsførelse kan søges igennem troen, men troen er samtidigt bevirket af Gud og omfatter kun nogle få.\footnote{Augustin, 2002, s. 811 (19,4)}

\subsubsection{De civ. Dei 21}
I De civ. Dei 21 redegør Augustin for Guds straf. Ifølge De civ. Dei 21,11 gør Augustin opmærksom på, at gengældelse spiller en stor rolle i retfærdighedens orden. Retfærdighed kommer i Augustins optik an på forholdet 'øje for øje, tand for tand'. Således er det for Augustin vigtigt at fastslå, at "den, der har gjort ondt, lider ondt."\footnote{Augustin, 2002, s. 916 (21,11)} I denne forbindelse er helvede kun til for straffens skyld, idet hans pointe er, at helvedes funktion er at rette op på uretten. I denne sammenhæng lægger Augustins vægt på ret livsførelse som frelseskriteriet, og at mennesket oplever det helvede, de for dem selv har skabt.

Den evige straf fungerer i Augustins tanker som en evig oplevelse af død.\footnote{Augustin, 2002, s. 913 (21,9)} Samtidig gradbøjes straffen efter, hvad den enkelte har fortjent.\footnote{Augustin, 2002, s. 921 (21,16)} I helvede oplever man ifølge Augustin den anden død som det, mennesker, der vender ryggen til Gud, fortjener. I De civ. Dei 21,12 gør Augustin samtidigt opmærksom på, at arvesynden skyldiggør alle til evig straf. Arvesynden er på den måde forudsætningen for, at Gud prædestinerer nogle til frelse og andre til dom. For mennesket havde som udgangspunkt sin glæde i Gud, men i gudløshed har det forladt Gud og gjort sig fortjent til "det evige onde, ved at ødelægge i sig det gode, som kunne have varet for evigt."\footnote{Augustin, Om Guds stad, s. 917 (21,12)}

Det er således vigtigt, at menneskeheden bliver opdelt. Fordi denne opdeling skal understrege, hvad den barmhjertige nåde og den retfærdige straf formåer. Ifølge Augustin er hævnens virkelighed til, for at retfærdigheden skal stå sin ret. På den måde understreger hævnens straf, hvad alle egentlig fortjener. Derimod understreger Augustin med hævnens virkelighed Guds uforskyldte gave i befrielses agten.\footnote{Augustin, 2002, s. 917 (21,12)}

\subsubsection{De civ. Dei 22}
De civ. Dei 22 omfatter Augustins forestilling om Guds stads evige salighed. Augustin reflekterer over synden som resultat af den frie vilje. Dog er denne vilje også forudsætningen for det gode, som Gud kan skabe ud af det onde. Samtidig gør Augustin opmærksom på, at Gud i sin nåde har skabt mennesket værdigt til himmelen, hvis bare det holder sig til Gud. Men samtidigt sker intet udenom Guds vilje, som ifølge De civ. Dei 22,2 har forudbestemt alle tings bevægelse mod deres udgange og formål.

\section{Regin Prenter}
Prenters forestillingen om den yderste dom bygger på bibelske skrifter og kirkens bekendelsesskrift Den Augsburgske Bekendelse (C.A.). Ifølge Prenter er teologi og kristologi det samme,\footnote{Prenter, 1971, s. 322} det vil sige, at læren om Kristus er åbenbaringsgrundlaget for læren om Gud. I modsætning til skabelsesteologien, hvor Gud kan erkendes igennem sin immanens, så mener Prenter, at mennesket kun har kendskab til Gud igennem Guds egen åbenbaring i tiden, nemlig i Kristus. Det vil sige, at Prenters teologiske tankegang omfatter, at Gud først og af fri vilje har besluttet sig for at give mennesket mulighed for at opnå frelse.

\subsubsection{Frelsen i kraft af fornyelse}
Prenters forestilling om menneskets frelse er centreret omkring en fornyelse. Dette udtryk er bærende for Prenters lære om frelsen, og får en særligt fremtrædende plads indenfor ekklesiologiens ramme. Forstået på den måde, at fornyelsen ifølge Prenter foregår i samarbejde med kirken.

Redegørelsen for fornyelsen omfatter fem former, der i Prenters optik alle står under et eskatologisk fortegn. Den kristne eskatologi forstår Prenter på baggrund af en forventning og et håb, der begge to er forankret i den historiske åbenbaringen i Jesus Kristus.\footnote{Prenter, 1971, s. 589} Fornyelsens fem former omfatter, at mennesket skal undergå en forvandling med opstandelsen for øje. Det vil sige, at forvandlingen igennem legemets død fuldendes i opstandelsen.\footnote{Prenter, 1971, s. 216} Forvandlingen tager udgangspunkt i en trosafgørelse, der gør mennesket opmærksom på, at det er retfærdiggjort på trods af, at det er en synder. Dette er baggrunden for Prenters opfattelse af 'simul justus et peccator', som siger, at mennesket på en gang er retfærdiggjort og synder. Sammenhængen i dette er, at Prenter forestiller sig mennesket som en enhed bestående af det gamle menneske, synderen, og det nye menneske, den retfærdiggjorte. Disse to dele af mennesket er i kamp med hinanden, om hvilken del af det skal komme til udtryk. Fornyelsens funktion, hvori Helligåndens virksomhed spiller en afgørende rolle som den fremaddrivende kraft, er igennem kampsituationen at påminde mennesket om, at det kæmper imod det, der står ånden, håbet og troen imod.\footnote{Prenter, 1971, s. 486}

Endemålet med fornyelsen er fuldendelsen, som betyder en delagtighed i Kristi opstandelse. Der er for Prenter en klar retning i fornyelsen, som er en tilblivelse imod fuldendelsen. Fuldendelsen er opnået, når troen, ved hjælp af Helligånden, har sejret over synden, og håbet er blevet bekræftet i opstandelsen. I opstandelsen bliver mennesket 'totos justus', det vil sige fuldstændig retfærdiggjort.\footnote{Prenter, 1971, s. 530}

Som allerede nævnt foregår fornyelsen i mennesket i samarbejde med kirken. Ifølge Prenter er kirken ud fra nytestamentlig og reformatorisk opfattelse det nye pagts gudsfolk. Kirken opfatter Prenter som 'Kristi legeme', og den instans, der forkynder ordet og forvalter sakramenterne på Kristi veje. Fornyelsen virkes kun af Helligånden i kraft af Guds forsoning og inkarnation i Jesus Kristus. Dette gør Prenter opmærksom på i artikel 5 i C. A, hvor Helligåndens gerning opfattes som den, der virker troen, samler Kirken og gør ord samt sakramente virkekraftigt.\footnote{Prenter, 1971, s. 482} Ifølge Prenter foreligger enheden mellem Faderen, sønnen og Helligånden som fundamentet for forsoningens og fornyelsens enhed.\footnote{Prenter, 1971, s. 483} Det vil sige, at frelsesgerningen, som Gud udretter i Jesus Kristus, bliver virkelighed i det enkelte menneske som frelsesvej, igennem Helligåndens værk samt nådemidlerne, forkyndelsen og sakramenterne, dåb og nadver.\footnote{Prenter, 1971, s. 557} Fornyelsen, som Helligåndens værk, er processen, hvorigennem mennesket bevæger sig hen mod frelsen.\footnote{For yderligere uddybning se Prenter, 1971, §24-39}

\subsubsection{Dommen}
Ifølge Prenter afhænger dommen af Kristi genkomst på baggrund af, at Menneskesønnen, ud fra bibelske
udsagn, forventes at komme igen som dommer. Prenter opfatter spørgsmålet om menneskets skæbne hinsides
den fysiske død at være skjult i den yderste dom. Grundlaget for dommen er ifølge Prenter forholdet til
Jesus Kristus. Derved kommer spørgsmål om tro til at spille en stor rolle. Det vil sige, dommen tager udgangspunkt i den enkeltes tro på Jesus eller manglen på det samme.\footnote{Prenter, 1971, s. 604} Dens funktion er at åbenbare menneskets skjulte liv og at anerkende Kristi herlighed. På den måde åbenbares der i dommen, hvor skillelinjen mellem de fortabte og de frelste går.\footnote{Prenter, 1971, s. 602}

Det, at dommen er skjult, er for Prenters en vigtig pointe i forhold til tanken om prædestination. Prenter opfatter prædestinationen som en gåde. Dette således, at frelsen kun kan søges via troen på Jesus Kristus, så længe gåden forbliver. Derved kan dommen holdes ved sit rette grundlag, i den forstand, at dommen er Guds anliggende.\footnote{Prenter, 1971, s. 605}

Prenters lære om dom omfatter også, at dommens udgang er dobbelt. Den ene udgang fører til frelse og den anden fører til fortabelse. Dette hænger tæt sammen med Prenter opfattelse af, at troen er grundlag for dommen. For, hvis frelsen beror på troen på Kristus, så må fortabelsen bero på at miste Kristus.\footnote{Prenter, 1971, s. 617} Prenter gør i samme omfang opmærksom på, at fortabelsen ligesom prædestinationen skal forstås som en gåde. Dette forstået således, at gåden i sig selv er en advarsel om, at mennesket skal holde sig til troen og stole på ordet.\footnote{Prenter, 1971, s. 619, for uddybelse se Matt. 8:12, 18:8-9, 23:33, 25:41 }