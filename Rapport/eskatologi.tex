\chapter{Eskatologi}
\label{chp:esk}
Eskatologi er læren om de sidste ting, men indenfor selve eskatologien findes der en variation af perspektiver. Temaer, man i eskatologien reflekterer over, er død, dom, opstandelse, himmel og helvede. Dette er temaer, der vedrører ethvert menneskes forestilling om dets livsmuligheder. Ovenfornævnte temaer vil blive præsenteret i følgende afsnit.

\section{Troen på himmelen}
Jerry L. Walls beskriver troen på himmelen som ækvivalent med troen på Gud. Teisme, troen på, at der findes en Gud, fordrer, at universet vidner om Guds herlighed som aftrykt i naturen. Aftrykket er blegt i forhold til den fulde udfoldelse af Guds herlighed, der overgår mennesket selv. Himmelens funktion er i denne forstand at efterlade mennesket i en stræben efter himmelen som opnåelsen af delagtighed i Guds herlighed. Derfor bliver forståelsen af himmelen afgørende i forhold til forståelsen af Gud. Her gør Walls opmærksom på, at kristendom i sin forestillingen om Gud udskiller sig fra andre religioner. Den skildrer nemlig en Gud, der i kraft af sin treenighed og kærlighed til mennesket møder mennesket som menneske. Derved muliggør Gud et forhold mellem mennesket og ham selv på grundlag af inkarnationen, døden og opstandelsen igennem Jesus Kristus. Walls konkludere ud fra dette, at himmelen som et kristent koncept således må forstås ud fra det bibelske narrativ af syndefaldet og retfærdiggørelsen.1 

Videre skriver Walls, at himmelen historisk set har fungeret som en kilde til moral. Moral er i denne forbindelse forstået som noget, der gør livet værd at leve.2 Denne forestilling har været under kritik, fordi der er bred mening om, at mennesket altid vil forsøge at skabe mening ud fra dets livsvilkår på den ene eller anden måde. Walls supplerer denne kritik med en refleksion over de kristnes håb. Håbet konstitueres af løftet om delagtighed i Kristi opstandelse, Guds kærlighed og Guds sejrer over døden. Himmelen forekommer i den forstand som et symbol på den ultimative gave fra Gud, hvis evige natur er selvhengivende kærlighed.3 Wells konkluderer således, at troen og håbet på himmelen resulterer i en fastholdelse af troen på en kærlig og almægtig Gud. Dette på trods af en virkelighed som udfordrer Guds tilværelse midt i ondskaben, også udtrykt teodicéproblemet.4

\section{Helvede som dom}
Jonathan L. Kvanvig introducerer tanken om helvede og skriver, at det har sin rod i den monoteistiske tradition under fælles betegnelsen de Abrahamistiske religioner. Kvanvig gør opmærksom på, at lærerne om himmel og helvede er tæt forbundne, og omfatter forestillinger om menneskets skæbne i efterlivet. Dertil skriver Kvanvig, at de har en social funktion, idet de motiverer tilhørere til god moralsk opførsel. Den traditionelle tilgang til helvede er, ifølge Kvanvig, at opfatte det som en straf.1 Der er fire traditioner, som er kendetegnende for forestillingen om helvedes straf:

\begin{itemize}
\item Helvedes formål er at straffe dem, der igennem sit jordiske liv har skyldiggjort sig til dommen i helvede.
\item Det er metafysisk umuligt at flygte fra helvede, når man først er blevet idømt helvede.
\item Nogle mennesker vil blive idømt helvede.
\item Man er bevidst om sin tilstedeværelse i helvede.
\end{itemize}

Doktrinen om helvede bliver forsvaret med, at alle forseelser er at begå uret mod Gud. Den traditionelle forestilling om helvede fastholder, at alle idømmes den samme straf. Dertil omfatter denne forestilling, at det er en uendelig ond handling at begå uret mod Gud, og derved retfærdiggør en forseelse mod Gud en uendelig straf. Kvanvig gør i denne forbindelse opmærksom på, at denne pointe er skrøbeligt. For mennesket har generelt ikke intentioner om at forse sig mod Gud. Skulle dette være tilfældet, så forholder det sig sjældent sådan, at mennesket er sig bevist om, at det begår en uret mod Gud. 

Ifølge Kvanvig findes der to modeller for forståelsen af helvede, straf-modellen og valg-modellen.1 Den traditionelle forståelse af helvede, straf-modellen, ser helvede som en fortjent staf. Derimod er den nyere forståelse af helvede, valg-modellen, baseret på, at den enkeltes tilstedeværelse i helvede er baseret på menneskets frie vilje.

\section{Troen på alles frelse}
Indenfor kristendommen er der en forestillingen kaldet frelsesuniversalisme, der har en baggrund i tanken om apokatastasis, som omfatter en genoprettelse af alle ting. Tanken om skærsilden og håbet om universel frelse kommer under denne forestillingskategori.

\subsubsection{Skærsilden}
Paul J. Griffeths redegør for forestillingen om skærsilden. Selve ordet skærsild er oversat fra det latinske ord purgatorium og betyder ’renselsessted’. Ifølge Griffeths kan skærsilden enten forestilles som et renselsessted eller en renselsestilstand. Man indgår i skærsilden ved døden, og befinder sig fra dette tidspunkt i en tilstand mellem døden og himmelen. Den eneste udgang fra skærsilden fører nemlig ind i himmelen. Videre gør Griffeths opmærksom på, at denne tilstand er midlertidig. Behovet for en renselse er, ifølge Griffeths, at det, der holder en separeret fra Guds kærlighed, skal renses væk. Når dette er opnået, kan man forlade skærsilden og komme ind i himmelen. Griffeths gør opmærksom på, at skærsilden strengt taget ikke hører ind under eskatologi. Dette fordi det ikke er blandt de sidste ting, men en midlertidig ting indtil man kommer ind i himmelen, som derimod er det sidste ting. 
Som nævnt ovenfor, så fører skærsildens eneste udgang ind i himmelen. Derfor mener Griffeths, at skærsilden skal opfattes som en del af himmelen. Afsluttende reflekterer han om Dantes forestilling om skærsilden, hvor skærsilden både er himmelens forkammer og det jordiske liv, og dens funktion er at transformere og klargøre mennesket til at møde Gud.

\subsubsection{Universalisme}
Thomas Talbott redegør i artiklen Universalism for tanken om universel frelse. Denne forestilling omfatter, at alle mennesker forsones med Gud samt det, at Gud genopretter hele skabelsen til sin oprindelige tilstand. Dette omfatter ifølge Talbott, at Gud indenfor kristen frelsesuniversalisme tilintetgør synd og død for at bringe evigt liv til alle. Indenfor frelsesuniversalisme er spørgsmålet om fri vilje et debatteret evne, fordi Gud i forsoningen underminerer menneskets livsvalg.  Talbott redegør for dette dilemma med eksempelet om S, der afviser Gud. S’s valg opfattes som irrationelt. Ydermere har S ikke muligheden for frit at afvise den sande Gud, således, at det S rent faktisk afviser er en karikatur af Gud. Trods mange fortalere for forståelsen af helvede som et valg er imod ovenfornævnte tanke, så kan mange ifølge Talbott acceptere præmissen, at en fri og fuldt informeret beslutning om at afvise Gud logisk er umulig. Tanken bag dette er, at en person, der afviser Gud, er aldrig fuldt informeret.  Det vil sige, at menneskets ulydighed i denne sammenhæng ifølge Talbott oftest optræder i en kontekst af tvetydighed, uvidenhed og illusion. I denne forbindelse spørger Talbott:  I hvilken forstand  mennesket kan holdes moralsk ansvarlig for dets tilblivelse? Her fremstillem Talbott en mulig forståelse af moralsk ansvarlighed, der omfatter, at den må afhænge af det pågældendes menneskes evne til at lære en moralsk lærestreg, og derved dets evne til at rette op på tingene. Dette medfører, at mennesket er undermineredt i sin frie vilje til at bestemme dets evige skæbne. Derimod er mennesket frit stillet til at afgøre, hvad det stadig har tilbage at tage lære af.  Afsluttende gør Talbott opmærksom på, at Gud på grund af hans evige kærlighed og visdom ved helligt forsyn kan styre menneskets liv således, at de ting, som det har brug for at lære, kommer mennesket i sidste ende til at lære.

\section{Tilintetgørelse som straf}
Clark H. Pinnock redegør for tanken om annihilation inden for kristen teologi i artiklen Annihilationism. Udtrykket annihilation kommer fra det latinske ord nihil og betyder ’intet’. Tilsvarende indebærer tanken om annihilation, at Guds yderste dom og straf for ondsindede mennesker er en tilintetgørelse af deres eksistens. Pinnock redegør for forestillingen om annihilation med bibelske skriftsteder. Pinnock bruger Jesu prædikener som eksempel og gør opmærksom på, at helvede bruges mere som advarsel om, at en eksklusion fra Gud medfører et evigt tab. Helvede er i disse eksempler ikke eksplicit beskrevet, men forekommer at fungere som billedsprog på en ødelæggelse eller udslettelse af eksistensen frem for evig pinsel. Pinnock underbygger teorien om annihilation med skriftsteder fra for eksempel Matt, 2. Thess, Gal. 1, 2. Pet, Rev. og Mal. Denne teori gør op med tanken om uendelig pinsel, men ikke med helvede som sådan. Pinnock gør opmærksom på, at tilintetgørelsen er en langt mere passende straf end evig straf. For ifølge ham må Guds straf for mennesket være kommensurable i det forhold, at finit synd kan ikke straffes med infinit straf.1 Pinnock mener i denne forbindelse, at Guds nåde ikke ville tillade mennesket at påføre sig en så uoprettelig skade som det at pines i evighed. Pinnock konkluderer, at helvede, rettelig forstået, udtrykker menneskets frie valg til at gøre godt eller ondt. Derved bliver helvede, ifølge Pinnock, ikke Guds valg, men menneskets valg. Ud fra Pinnocks redegørelse af annihilationsteorien, fastholder teorien helvede uden at fastholde pinsels-motiverne. Helvede er i den forstand menneskets mulighed til at bruge sin Guds-givne frihed til at miste Gud og tilintetgøre sig selv.2

