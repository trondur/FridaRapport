\chapter{Skærsilden}
\textit{Den kateketiske tale} og \textit{Om sjælen og opstandelsen} er dogmatiske skrifter af Gregor af Nyssa. I disse skrifter gør Gregor rede for sin forestilling om skærsilden. Tanken med skærsilden omfatter, at den menneskelige sjæl skal igennem en rensende ild for at opnå endelig frelse. Det essentielle ved idéen om renselsen er, at noget skal tilintetgøres for at nyskabelsen kan vokse frem i det gamles sted.

Det centrale for Gregors forestilling er, at han opfatter mennesket som værende født med en frihed men også med en indre længsel. Denne længsel er en bevægelse i sjælen mod Gud. Dette forstået således, at mennesket, på grund af sjælen, kender til Guds eksistens.1 Dog er dette ikke ensbetydende med at vide, hvad Gud er. For for mennesket er han ubegribelig. Derfor bliver menneskets bevægelse mod Gud en stadig udgriben efter Gud. Dette skal forstås som, at mennesket, ved at se på verdenen, i sig selv danner et billede af, hvad Gud måtte være. Men billedet kommer altid at være en håndsudrækkelse mod Gud, hvor man aldrig helt når op til Gud.

Essentielt for Gregors opfattelse er, legemets fornuft, at sjælen er skabt og det, der bistår legemet i dets vitalitet.2 Ydermere forstår Gregor evnen til at tænke og overveje som et guddommeligt karaktertræk. Det vil sige, at sjælen er beslægtet med Gud idet, at den bærer efterligninger til det guddommelige i form af en genspejling.3 Det er nemlig på grund af det guddommelige aftryk i sjælen, at mennesket har en eksistentiel længsel mod det guddommelige.4 Sjælen har ifølge Gregor taget bo i legemet, men på grund af dens slægtskab med Gud, så er den forskellig fra legemets faste stof.

\subsubsection{Opstandelsen i lyset af døden}
Gregors opfattelse af opstandelsen skal ses i lyset af døden. Ifølge ham vil alle mennesker stå op fra de døde -- nogle til dom og andre til frelse. Der er derfor ingen grund til at frygte døden, hvis man kender til dens betydning. For Gregor er døden nærmest en gave, hvis den rettelig forstås i lyset af opstandelsen. Med reference til 1. Mos. 2:21 redegør Gregor for, at dengang Gud iklædte de første mennesker i skind, gav han dem osgå egenskaben til at dø. Dette forstået således, at klæderne eller rettere dødeligheden omslutter mennesket udefra, men dets indre guddommelige natur er trods dødelighedens kåbe intakt.5 Dødens funktion markerer i denne forbindelse en overgivelse af mennesket til udødeligheden, idet Gud ”gennem opstandelsen [former] en krukke på ny, der er renset for det onde ved hjælp af opstandelsen.”6 Gregor ser derfor opstandelsen som et bindeled mellem legeme og sjæl, og kalder denne forening 'mysteriet i Guds frelsesplan'. Foreningen mellem legeme og sjæl symboliserer mødestedet for liv og død, hvor Gud og mennesket kan forenes med hinanden.7 Døden bliver på en måde et redskab for Gud at ophæve synden. Således forstået, at døden i sig selv er negativ indtil Kristus forener død og liv, og derved muliggør den treenige økonomis forening.

\subsubsection{Den rensende ild}
Skærsilden har en vigtig funktion for foreningen mellem menneske og Gud. Mennesket er som før nævnt en sammensat enhed. Ifølge Gregor er det når sjælen kommer i kontakt med det legemlige liv igennem sanserne, at sjælen kan tilsmudses. Dette er tilfældet, hvis fornuften mister magten over lidenskaberne, hvilket resulterer i, at mennesket synker fra det fornuftige og guddommelige til det ufornuftige og dåreagtige.8 Lidenskaber sætter sig i så fald fast på sjælen og forårsager, at den gøres mere stoflig.9 Resultatet er en renselse af sjælen, som medfører en erfaring af store smerter ”når den guddommelige kraft af kærlighed til mennesket trækker det, der er dens eget, ud af de ufornuftige og stoflige ruiner.”10 Tilbage står det ufuldkomne i mennesket, skidtet, som ifølge Gregor skal renses, fortæres og tilintetgøres ved ild. Gregor forstår skærsilden som Guds kærlighed til mennesket, fordi denne muliggør alles frelse. Skærsilden som dom er derfor ikke Guds måde at påføre lidelse, men at skille godt fra det onde.

Renselse og frelse er ifølge Gregor Guds eneste mål. Det vil sige, at alle, når menneskets naturs fylde er blevet fuldkommen, vil få del i det gode som findes i Gud.11 Dette at få del i det gode opfatter Gregor at være i Gud. Videre anfører Gregor, at opstandelsen har den betydning, at menneskets natur vender tilbage til dens oprindelige tilstand. Denne opfattelse tager udgangspunkt apokatastasislæren. Ifølge Gregor genvinder alle den oprindelige fuldkommenhed og denne genoprettelsen opfatter han som en helbredelse. Helbredelsen omfatter, at det, der har blandet sig med sjælen, renses og aflægges.12 Idet sjælen besidder bevidstheden er det ifølge Gregor kun sjælen, der kan erfare døden idet legemet dør. Straffens form ifølge Gregors er transformeret til kun at omfatte en renselse, igennem hvilken Gud drager mennesket til sig og omformer sjælen til at omfatte dens oprindelige renhed.