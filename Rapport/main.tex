\documentclass[12pt,a4paper,titlepage]{report} 
\usepackage{titlesec}
\usepackage{cite}
\usepackage{listings}
\usepackage{amsmath}
\usepackage{calligra}
\usepackage[T1]{fontenc}
\lstset{language=csh, frame=single, basicstyle=\footnotesize}
\usepackage{float} %enables use of [H] placement specifiers in figures 
\usepackage[utf8]{inputenc}
\usepackage{hyperref}
\usepackage{array}
\usepackage[table, x11names]{xcolor}
\usepackage{svg}
\usepackage{graphicx}
\usepackage{placeins}
\usepackage{longtable}
\usepackage{morefloats}
\usepackage{listings}
\usepackage{color}
\renewcommand{\contentsname}{Indholdsfortegnelse}
\renewcommand*\listfigurename{Figures}

\newcommand{\tab}[1]{\hspace{1cm}{#1}}

\titleformat{\chapter}
{\normalfont\LARGE\bfseries}{\thechapter.\ }{10pt}{}
\titlespacing*{\chapter}{0pt}{0pt}{10pt}

\begin{document}
\begin{titlepage}
\newcommand{\HRule}{\rule{\linewidth}{0.5mm}} % Defines a new command for the horizontal lines, change thickness here

\center % Center everything on the page
 
%----------------------------------------------------------------------------------------
%	HEADING SECTIONS
%----------------------------------------------------------------------------------------

%	LOGO SECTION
%----------------------------------------------------------------------------------------
\includegraphics[width=0.5\textwidth]{logo.jpg}\\[1cm] % Include a department/university logo - this will require the graphicx package
%----------------------------------------------------------------------------------------

\vspace{1 cm}
\textsc{\LARGE Bachelor Projekt} \\[0.4cm] % Minor heading such as course title
\textsc{\large Systematisk teologi i dogmatik} \\
%----------------------------------------------------------------------------------------
%	TITLE SECTION
%----------------------------------------------------------------------------------------
\vspace{1 cm}
\HRule \\[1.0cm]
{ \LARGE \textbf{Hvordan påvirker forestillinger om dommedag tanken om livet og døden?} \small \vspace{0.4cm}}\\ % Title of your document
\HRule \\[1.0cm]
 
\center

\LARGE Frida Marjunardottir Thomsen \\
\vspace{0.2 cm}
\small fridamarjunardottir@gmail.com \\

\vspace{0.5 cm}

\large \textbf{Vejleder:} \\
\large Niels Henrik Gregersen\\
\vspace{0.4 cm}
{\large \today}\\[3cm] % Date, change the \today to a set date if you want to be precise
\end{titlepage}


\begin{abstract}
As humans we wonder about life and death in relation to, if there could be an afterlife. Many ideas about the eschatological events including divine judgment have evolved through the history of Christianity. Still today there has been no unity in the Christian circles about a doctrine concerning how the afterlife will be. However, there is a doctrine concerning the resurrection and this states that Christianity believes that there is an afterlife however it may appear.

The doctrine of double outcome, doctrine of apokatastasis and annihilationism are ideas concerning eschatology and all provide different ideas about the human situation inbetween life, death and afterlife. Since we humans all are going to die at some point, the thoughts about an afterlife and continuance from this life are enticing life opportunities. In this respect an analysis of Augustin, Regin Prenter, Jürgen Moltmann, Gregorory of Nyssa and Barth will present how they imagine the human situation in the light of divine judgment. However these theologians all encounter a point in the imagination of the divine where they have to acknowledge that some things demand ones silence and rely on a hope for atonement.
\end{abstract}
\setcounter{section}{0}
\bibliographystyle{apalike}



\setcounter{tocdepth}{1}
{\small \tableofcontents}

\clearpage

\chapter{Introduktion}
Hvad bevæger vores liv sig hele tiden hen imod er et eksistentialistisk funderet spørgsmål. Kristendommen svar på dette spørgsmål er læren om de dødes opstandelse. Denne lære kommer under eskatologi, som betyder læren om de sidste tider og omhandler ethvert menneskes skæbne som død, dom, opstandelse, himmel og helvede. Indenfor kristendommen kan man finde tre teologiske hovedsynspunkter indenfor eskatologi: læren om dobbelt udgang, apokatastasislæren og annihilationsteorien. Disse tre synspunkter er forskellige opfattelser af menneskets sjælstilstand, da skiftet mellem liv og død indtræffer.

Som sagt omhandler eskatologi ethvert menneskes skæbne, men derudover omhandler den også menneskets håb på en kontinuitet efter dødens indtrædelse. Omdrejningspunktet for kristen eskatologi er Kristi genkomst, hvor han vil dømme levende og døde. Denne begivenhed er genstand for menneskets tro og håb på Guds frelse. I den forstand er dommen konstituerende for troen, idet den fungerer som et spejl og holder mennesket medansvarligt i livet.1 Spørgsmålet, hvad der forestilles ved dommen, har op igennem tiderne ændret karakter. De forskellige traditioner at tænke om dommen varierer i forhold til, hvordan Gud vælger at møde mennesket i sin nåde. Opgaven fremstiller Augustins, Regin Prenters, Gregor af Nyssas, Jürgen Moltmanns og Karl Barths (skildret af Niels Henrik Gregersens) forestillinger om den yderste dom. Med baggrund i deres forestillinger vil opgaven søge at opstille, hvad der er muligt at sige om de sidste tider. Endvidere vil der i opgaven, med ovenfornævnte forestillinger som baggrund, være en refleksion over, hvordan menneskets tanker om de sidste tider påvirker forståelsen af livet.
\chapter{Tese}
De eskatologiske begivenheder ligger udenfor menneskets erfaringsevne, men det holder ikke mennesket fra at tænke over det. Spørgsmålet er, hvilken mening det giver at tænke ud over de grænser, hvor vores sanser ikke rækker? Forestillingerne forbliver ikke andet end forestillinger. Erfaringsgrænsen møder dog mennesket med en guddommelig stilhed\footnote{Udtryk lånt fra George Hunsinger: Hellfire and Damnation Four Ancient and Modern views}, som tillader et hav af fortolkninger –- i kristendommens tilfælde dogmet om de dødes opstandelse. Dette dogme lærer, at der er en opstandelse, men overfor den efterfølgende evigheds hvordan og hvorledes, har kristendommen forholdt sig tøvende vedrørende en definitiv konklusion. 

Emnet om Guds nåde, og hvordan denne kommer til udtryk, synes at være punktet, hvor teologerne rammer en mur af guddommelig stilhed. Spørgsmålet er om denne guddommelige stilhed ikke forholder sig stille som udtryk for accept af menneskets ret til dets egne forståelse af livet og det, der følger efter døden?

Alle mennesket skal med sikkerhed dø på et tidspunkt, men hvad der kommer til at ske med os efter døden ved vi ikke. Derfor kommer vores tanker om døden utvivlsomt til at påvirke vores liv.
\chapter{Kilder}
\label{chp:kilder}
Kilder til forståelse af Augustins tanke om en prædestineret dobbelt udgang er baseret på hans egne værker Enchiridion (Ench.) og De civitate Dei (De civ. Dei). Yderlig viden omkring Augustins teologi er hentet fra Anders-Christian Jacobsens artikel Augustin om menneskets opstandelse og Georg Hunsingers artikel Hellfire and Damnation: Four Ancient and Modern views. Kildematerialet til forståelse af Regin Prenters opfattelse af prædestineret dobbelt udgang er baseret på Prenters eget dogmatiske værk Skabelse og Genløsning. Til fremstilling af Jürgen Moltmanns opfattelse af frelsesuniversalisme har hans eskatologiske værker Theology of Hope og The coming of God fungeret som kildemateriale. Kildemateriale til forståelse af Gregor af Nyssas tanker om alles frelse er hentet fra bogen Gregor af Nyssa. Bogen indeholder oversættelser af tre af hans værker Markinas Liv, Om sjælen og Opstandelse og Den Kateketiske Tale. Hele samlingen har fungeret som baggrunds viden om Gregor. Artiklen Barth og annihilationsteorien skrevet af Niels Henrik Gregersen er eneste kildegrundlag til fremstilling af Barths tanke om dobbelt udgang og hans udvælgelseslære. The Oxford Handbook of Eschatology fungerer som kilde til afsnit \ref{chp:esk} Eskatologi, som er en forklarende oversigt over emner, der har med forestillinger om himmel, helvede, frelsesuniversalisme og annihilation at gøre.
\chapter{Teori og metode}
Forestilling som teologerne nævnt i afsnit \ref{chp:kilder} har om de eskatologiske begivenheder fungerer som baggrund for teorien om kristendommens eskatologi, og hvad den omfatter. Begreberne dobbelt udgang, apokatastasis og annihilation er centrale for forståelsen af pågældende teologers teser om, hvad eskatologien omfatter. Desuden specificerer nævnte begreber, hvordan guds nåde og frelse kommer til udtryk.

Selve opgaven forholder sig systematisk analytisk til forestillingerne om dom i deres varierende former og fremstiller dem hver for sig. Hver enkelt af fremstillingerne er afgrænset i sin redegørelse og forholder sig til tanken om den yderste dom. Indledningsvis er der en forklarende fremstilling af kristen eskatologi baseret på The Oxford Handbook of Eschatology. Afsnit ”REFREFREF”Diskussion indeholder en komparativ analyse af forestillingerne. Analysen opstiller sammenhængen af det fokus forestillingerne har på liv, død og efterliv. Denne komparative analyse suppleres af en personlig refleksion over forestillingernes fokus sat i forhold til mennesket.
\chapter{Eskatologi}
\label{chp:esk}
Eskatologi er læren om de sidste ting, men indenfor selve eskatologien findes der en variation af perspektiver. Temaer, man i eskatologien reflekterer over, er død, dom, opstandelse, himmel og helvede. Dette er temaer, der vedrører ethvert menneskes forestilling om dets livsmuligheder. Ovenfornævnte temaer vil blive præsenteret i følgende afsnit.

\section{Troen på himmelen}
Jerry L. Walls beskriver troen på himmelen som ækvivalent med troen på Gud. Teisme, troen på, at der findes en Gud, fordrer, at universet vidner om Guds herlighed som aftrykt i naturen. Aftrykket er blegt i forhold til den fulde udfoldelse af Guds herlighed, der overgår mennesket selv. Himmelens funktion er i denne forstand at efterlade mennesket i en stræben efter himmelen som opnåelsen af delagtighed i Guds herlighed. Derfor bliver forståelsen af himmelen afgørende i forhold til forståelsen af Gud. Her gør Walls opmærksom på, at kristendom i sin forestillingen om Gud udskiller sig fra andre religioner. Den skildrer nemlig en Gud, der i kraft af sin treenighed og kærlighed til mennesket møder mennesket som menneske. Derved muliggør Gud et forhold mellem mennesket og ham selv på grundlag af inkarnationen, døden og opstandelsen igennem Jesus Kristus. Walls konkludere ud fra dette, at himmelen som et kristent koncept således må forstås ud fra det bibelske narrativ af syndefaldet og retfærdiggørelsen.1 

Videre skriver Walls, at himmelen historisk set har fungeret som en kilde til moral. Moral er i denne forbindelse forstået som noget, der gør livet værd at leve.2 Denne forestilling har været under kritik, fordi der er bred mening om, at mennesket altid vil forsøge at skabe mening ud fra dets livsvilkår på den ene eller anden måde. Walls supplerer denne kritik med en refleksion over de kristnes håb. Håbet konstitueres af løftet om delagtighed i Kristi opstandelse, Guds kærlighed og Guds sejrer over døden. Himmelen forekommer i den forstand som et symbol på den ultimative gave fra Gud, hvis evige natur er selvhengivende kærlighed.3 Wells konkluderer således, at troen og håbet på himmelen resulterer i en fastholdelse af troen på en kærlig og almægtig Gud. Dette på trods af en virkelighed som udfordrer Guds tilværelse midt i ondskaben, også udtrykt teodicéproblemet.4

\section{Helvede som dom}
Jonathan L. Kvanvig introducerer tanken om helvede og skriver, at det har sin rod i den monoteistiske tradition under fælles betegnelsen de Abrahamistiske religioner. Kvanvig gør opmærksom på, at lærerne om himmel og helvede er tæt forbundne, og omfatter forestillinger om menneskets skæbne i efterlivet. Dertil skriver Kvanvig, at de har en social funktion, idet de motiverer tilhørere til god moralsk opførsel. Den traditionelle tilgang til helvede er, ifølge Kvanvig, at opfatte det som en straf.1 Der er fire traditioner, som er kendetegnende for forestillingen om helvedes straf:

\begin{itemize}
\item Helvedes formål er at straffe dem, der igennem sit jordiske liv har skyldiggjort sig til dommen i helvede.
\item Det er metafysisk umuligt at flygte fra helvede, når man først er blevet idømt helvede.
\item Nogle mennesker vil blive idømt helvede.
\item Man er bevidst om sin tilstedeværelse i helvede.
\end{itemize}

Doktrinen om helvede bliver forsvaret med, at alle forseelser er at begå uret mod Gud. Den traditionelle forestilling om helvede fastholder, at alle idømmes den samme straf. Dertil omfatter denne forestilling, at det er en uendelig ond handling at begå uret mod Gud, og derved retfærdiggør en forseelse mod Gud en uendelig straf. Kvanvig gør i denne forbindelse opmærksom på, at denne pointe er skrøbeligt. For mennesket har generelt ikke intentioner om at forse sig mod Gud. Skulle dette være tilfældet, så forholder det sig sjældent sådan, at mennesket er sig bevist om, at det begår en uret mod Gud. 

Ifølge Kvanvig findes der to modeller for forståelsen af helvede, straf-modellen og valg-modellen.1 Den traditionelle forståelse af helvede, straf-modellen, ser helvede som en fortjent staf. Derimod er den nyere forståelse af helvede, valg-modellen, baseret på, at den enkeltes tilstedeværelse i helvede er baseret på menneskets frie vilje.

\section{Troen på alles frelse}
Indenfor kristendommen er der en forestillingen kaldet frelsesuniversalisme, der har en baggrund i tanken om apokatastasis, som omfatter en genoprettelse af alle ting. Tanken om skærsilden og håbet om universel frelse kommer under denne forestillingskategori.

\subsubsection{Skærsilden}
Paul J. Griffeths redegør for forestillingen om skærsilden. Selve ordet skærsild er oversat fra det latinske ord purgatorium og betyder ’renselsessted’. Ifølge Griffeths kan skærsilden enten forestilles som et renselsessted eller en renselsestilstand. Man indgår i skærsilden ved døden, og befinder sig fra dette tidspunkt i en tilstand mellem døden og himmelen. Den eneste udgang fra skærsilden fører nemlig ind i himmelen. Videre gør Griffeths opmærksom på, at denne tilstand er midlertidig. Behovet for en renselse er, ifølge Griffeths, at det, der holder en separeret fra Guds kærlighed, skal renses væk. Når dette er opnået, kan man forlade skærsilden og komme ind i himmelen. Griffeths gør opmærksom på, at skærsilden strengt taget ikke hører ind under eskatologi. Dette fordi det ikke er blandt de sidste ting, men en midlertidig ting indtil man kommer ind i himmelen, som derimod er det sidste ting. 
Som nævnt ovenfor, så fører skærsildens eneste udgang ind i himmelen. Derfor mener Griffeths, at skærsilden skal opfattes som en del af himmelen. Afsluttende reflekterer han om Dantes forestilling om skærsilden, hvor skærsilden både er himmelens forkammer og det jordiske liv, og dens funktion er at transformere og klargøre mennesket til at møde Gud.

\subsubsection{Universalisme}
Tanken om universel frelse redegør Thomas Talbott for. Denne forestilling omfatter at alle mennesker vil blive forsonet samt at hele universet oplever en genoprettelse til sin oprindelige tilstand. Talbott redegør for, at kristen frelsesuniversalisme omfatter at Gud tilintetgør synd og død for at bringe evigt liv til alle.

\section{Tilintetgørelse som straf}
Clark H. Pinnock redegør for tanken om annihilation inden for kristen teologi i artiklen Annihilationism. Udtrykket annihilation kommer fra det latinske ord nihil og betyder ’intet’. Tilsvarende indebærer tanken om annihilation, at Guds yderste dom og straf for ondsindede mennesker er en tilintetgørelse af deres eksistens. Pinnock redegør for forestillingen om annihilation med bibelske skriftsteder. Pinnock bruger Jesu prædikener som eksempel og gør opmærksom på, at helvede bruges mere som advarsel om, at en eksklusion fra Gud medfører et evigt tab. Helvede er i disse eksempler ikke eksplicit beskrevet, men forekommer at fungere som billedsprog på en ødelæggelse eller udslettelse af eksistensen frem for evig pinsel. Pinnock underbygger teorien om annihilation med skriftsteder fra for eksempel Matt, 2. Thess, Gal. 1, 2. Pet, Rev. og Mal. Denne teori gør op med tanken om uendelig pinsel, men ikke med helvede som sådan. Pinnock gør opmærksom på, at tilintetgørelsen er en langt mere passende straf end evig straf. For ifølge ham må Guds straf for mennesket være kommensurable i det forhold, at finit synd kan ikke straffes med infinit straf.1 Pinnock mener i denne forbindelse, at Guds nåde ikke ville tillade mennesket at påføre sig en så uoprettelig skade som det at pines i evighed. Pinnock konkluderer, at helvede, rettelig forstået, udtrykker menneskets frie valg til at gøre godt eller ondt. Derved bliver helvede, ifølge Pinnock, ikke Guds valg, men menneskets valg. Ud fra Pinnocks redegørelse af annihilationsteorien, fastholder teorien helvede uden at fastholde pinsels-motiverne. Helvede er i den forstand menneskets mulighed til at bruge sin Guds-givne frihed til at miste Gud og tilintetgøre sig selv.2


\chapter{Læren om dobbelt udgang}
Forestillingen om dobbelt udgang omfatter, at der for mennesket ved dommedag foreligger to udgange med hver sin specifikke destination. Fortalere for denne opfattelse er Regin Prenter og Augustin. Ifølge dem afhænger destinationen af en dom, der er afgørende for hvilken udgang, det enkelte menneske idømmes. Den ene udgang fører til himmelsk frelse, mens den anden fører til evig fortabelse. Fælles for Prenters og Augustins opfattelse er, at Gud har prædestineret udgangen som dobbelt. Hvor de forekommer at være uenige er derimod opfattelsen af Guds nåde i forhold til Guds udvalgte.

\section{Augustin}
I Augustins Ench. og De civ. Dei. 19 – 22 fremgår forestillingen om menneskehedens absolutte skæbne som opdelt i to grupper, de frelste og de fortabte. Baggrunden for denne opfattelse er blandt andet at finde i Augustins refleksion over det ondes eksistens. Dette spørgsmål optager Augustin i stor grad, og præger hans opfattelse af Guds nåde, frelse og dom.

Augustins eskatologiske tanker er præget af skabelsesteologiske motiver. Ifølge ham er mennesket i besiddelse af en naturlig gudserkendelse. Det vil sige, at mennesket, igennem det skabte, kan erkende sin skaber, Gud. Dette synspunkt fremgår i De civ. Dei 22,24. Da syndefaldet indtraf skabte dette splid i naturen, som medførte en uoverensstemmelse imellem den himmelske og jordiske stad. Dette splid forårsagte, at mennesket ikke kunne erkende Gud, uden at Gud i sin nåde hjalp mennesket med at erkende Gud. Dette er ifølge Augustin årsagen til det ondes mulighed for at eksistere. 

Ud fra Augustins opfattelse af Gud forvalter Gud alt, som han selv har skabt. Ud fra dette perspektiv sætter Augustins spørgsmålstegn ved synden, som ifølge hans forestilling er af det onde. Ifølge Augustins anskuelse er Guds vilje uovertruffen, og kan derfor ikke være ond.  Hvordan kan det onde så eksistere? Augustins konklusion er, at det onde ikke kommer fra Gud, men at det eksisterer, fordi Gud tillader det onde at eksistere.\footnote{Augustin, 2002, s. 910 (21,7)} I Ench. forklarer Augustin således, at når Gud tillader det onde at ske, er det af retfærdighed. Derfor kommer det onde i Augustins optik til at være velanbragt på sit sted som forudsætningen for, at Gud i hans almægtighed kan gøre godt med det onde.\footnote{Augustin, 2005, s. 65 (3.11) } På grundlag af denne opfattelse konkluderer Augustin, at Gud tillod det onde at vinde indpas gennem syndefaldet. Følgen heraf var, at menneskets vilje blev beskadiget, og at mennesket ikke længere ville det gode, uden at Gud i sin barmhjertighed og nåde hjalp mennesket til at ville det gode. Da det efter Augustins forestilling er Gud, der bevirker vores vilje, stræben og tro, så bliver ingen frelst, uden at Gud vil frelse dem. Dette, at Gud bevirker aktiviteten i mennesket, er baggrunden for Augustins forståelse af Guds nåde som ufortjent. Som Prenter har formuleret det, så er Augustins nådeslære ufortjent, men ikke ufortjenstfuld. Mennesket er derfor prisgivet Gud, for ifølge Augustin afhænger ens frelse af, at Gud vil ens frelse.

\subsubsection{De civ. Dei 20}
De civ. Dei 20 omfatter Augustins forestilling om den yderste dag. Ifølge ham er den yderste dag regnskabets dag, hvor der opgøreles, hvem har fortjent lykke eller ulykke. Afgørende for dommens dag er Kristi genkomst, da det er ham der skal dømme levende og døde.\footnote{Augustin, 2002, s. 846 (20,1)} Dommens dag åbenbarer, hvem de gode og slette er, fordi de vil hver især blive dømt i forhold til det de fortjener.\footnote{Augustin, 2002, s. 946 (20,1)} Netop dette 'hvad hver enkelt fortjener' er centralt i forhold til Augustins tanke om nåden. Nåden i denne sammenhæng kommer at fremstå som et retsmæssig forhold mellem Gud og mennesket. Hovedfokus i dette forhold er straf og belønning og kommer i store træk til at afgøre, hvem Guds frelse omfatter. Ifølge Augustin er Guds frelsesvilje begrænset til kun at omfatte nogle få udvalgte. Augustin lægger vægt på ret livsførelse som forudsætning for opnåelsen af frelse og det evige liv.\footnote{Augustin, 2002, s. 811 (19,4)} Ret livsførelse kan søges igennem troen, men troen er samtidigt bevirket af Gud og omfatter kun nogle få.\footnote{Augustin, 2002, s. 811 (19,4)}

\subsubsection{De civ. Dei 21}
I De civ. Dei 21 redegør Augustin for Guds straf. Ifølge De civ. Dei 21,11 gør Augustin opmærksom på, at gengældelse spiller en stor rolle i retfærdighedens orden. Retfærdighed kommer i Augustins optik an på forholdet 'øje for øje, tand for tand'. Således er det for Augustin vigtigt at fastslå, at "den, der har gjort ondt, lider ondt."\footnote{Augustin, 2002, s. 916 (21,11)} I denne forbindelse er helvede kun til for straffens skyld, idet hans pointe er, at helvedes funktion er at rette op på uretten. I denne sammenhæng lægger Augustins vægt på ret livsførelse som frelseskriteriet, og at mennesket oplever det helvede, de for dem selv har skabt.

Den evige straf fungerer i Augustins tanker som en evig oplevelse af død.\footnote{Augustin, 2002, s. 913 (21,9)} Samtidig gradbøjes straffen efter, hvad den enkelte har fortjent.\footnote{Augustin, 2002, s. 921 (21,16)} I helvede oplever man ifølge Augustin den anden død som det, mennesker, der vender ryggen til Gud, fortjener. I De civ. Dei 21,12 gør Augustin samtidigt opmærksom på, at arvesynden skyldiggør alle til evig straf. Arvesynden er på den måde forudsætningen for, at Gud prædestinerer nogle til frelse og andre til dom. For mennesket havde som udgangspunkt sin glæde i Gud, men i gudløshed har det forladt Gud og gjort sig fortjent til "det evige onde, ved at ødelægge i sig det gode, som kunne have varet for evigt."\footnote{Augustin, Om Guds stad, s. 917 (21,12)}

Det er således vigtigt, at menneskeheden bliver opdelt. Fordi denne opdeling skal understrege, hvad den barmhjertige nåde og den retfærdige straf formåer. Ifølge Augustin er hævnens virkelighed til, for at retfærdigheden skal stå sin ret. På den måde understreger hævnens straf, hvad alle egentlig fortjener. Derimod understreger Augustin med hævnens virkelighed Guds uforskyldte gave i befrielses agten.\footnote{Augustin, 2002, s. 917 (21,12)}

\subsubsection{De civ. Dei 22}
De civ. Dei 22 omfatter Augustins forestilling om Guds stads evige salighed. Augustin reflekterer over synden som resultat af den frie vilje. Dog er denne vilje også forudsætningen for det gode, som Gud kan skabe ud af det onde. Samtidig gør Augustin opmærksom på, at Gud i sin nåde har skabt mennesket værdigt til himmelen, hvis bare det holder sig til Gud. Men samtidigt sker intet udenom Guds vilje, som ifølge De civ. Dei 22,2 har forudbestemt alle tings bevægelse mod deres udgange og formål.

\section{Regin Prenter}
Prenters forestillingen om den yderste dom bygger på bibelske skrifter og kirkens bekendelsesskrift Den Augsburgske Bekendelse (C.A.). Ifølge Prenter er teologi og kristologi det samme,\footnote{Prenter, 1971, s. 322} det vil sige, at læren om Kristus er åbenbaringsgrundlaget for læren om Gud. I modsætning til skabelsesteologien, hvor Gud kan erkendes igennem sin immanens, så mener Prenter, at mennesket kun har kendskab til Gud igennem Guds egen åbenbaring i tiden, nemlig i Kristus. Det vil sige, at Prenters teologiske tankegang omfatter, at Gud først og af fri vilje har besluttet sig for at give mennesket mulighed for at opnå frelse.

\subsubsection{Frelsen i kraft af fornyelse}
Prenters forestilling om menneskets frelse er centreret omkring en fornyelse. Dette udtryk er bærende for Prenters lære om frelsen, og får en særligt fremtrædende plads indenfor ekklesiologiens ramme. Forstået på den måde, at fornyelsen ifølge Prenter foregår i samarbejde med kirken.

Redegørelsen for fornyelsen omfatter fem former, der i Prenters optik alle står under et eskatologisk fortegn. Den kristne eskatologi forstår Prenter på baggrund af en forventning og et håb, der begge to er forankret i den historiske åbenbaringen i Jesus Kristus.\footnote{Prenter, 1971, s. 589} Fornyelsens fem former omfatter, at mennesket skal undergå en forvandling med opstandelsen for øje. Det vil sige, at forvandlingen igennem legemets død fuldendes i opstandelsen.\footnote{Prenter, 1971, s. 216} Forvandlingen tager udgangspunkt i en trosafgørelse, der gør mennesket opmærksom på, at det er retfærdiggjort på trods af, at det er en synder. Dette er baggrunden for Prenters opfattelse af 'simul justus et peccator', som siger, at mennesket på en gang er retfærdiggjort og synder. Sammenhængen i dette er, at Prenter forestiller sig mennesket som en enhed bestående af det gamle menneske, synderen, og det nye menneske, den retfærdiggjorte. Disse to dele af mennesket er i kamp med hinanden, om hvilken del af det skal komme til udtryk. Fornyelsens funktion, hvori Helligåndens virksomhed spiller en afgørende rolle som den fremaddrivende kraft, er igennem kampsituationen at påminde mennesket om, at det kæmper imod det, der står ånden, håbet og troen imod.\footnote{Prenter, 1971, s. 486}

Endemålet med fornyelsen er fuldendelsen, som betyder en delagtighed i Kristi opstandelse. Der er for Prenter en klar retning i fornyelsen, som er en tilblivelse imod fuldendelsen. Fuldendelsen er opnået, når troen, ved hjælp af Helligånden, har sejret over synden, og håbet er blevet bekræftet i opstandelsen. I opstandelsen bliver mennesket 'totos justus', det vil sige fuldstændig retfærdiggjort.\footnote{Prenter, 1971, s. 530}

Som allerede nævnt foregår fornyelsen i mennesket i samarbejde med kirken. Ifølge Prenter er kirken ud fra nytestamentlig og reformatorisk opfattelse det nye pagts gudsfolk. Kirken opfatter Prenter som 'Kristi legeme', og den instans, der forkynder ordet og forvalter sakramenterne på Kristi veje. Fornyelsen virkes kun af Helligånden i kraft af Guds forsoning og inkarnation i Jesus Kristus. Dette gør Prenter opmærksom på i artikel 5 i C. A, hvor Helligåndens gerning opfattes som den, der virker troen, samler Kirken og gør ord samt sakramente virkekraftigt.\footnote{Prenter, 1971, s. 482} Ifølge Prenter foreligger enheden mellem Faderen, sønnen og Helligånden som fundamentet for forsoningens og fornyelsens enhed.\footnote{Prenter, 1971, s. 483} Det vil sige, at frelsesgerningen, som Gud udretter i Jesus Kristus, bliver virkelighed i det enkelte menneske som frelsesvej, igennem Helligåndens værk samt nådemidlerne, forkyndelsen og sakramenterne, dåb og nadver.\footnote{Prenter, 1971, s. 557} Fornyelsen, som Helligåndens værk, er processen, hvorigennem mennesket bevæger sig hen mod frelsen.\footnote{For yderligere uddybning se Prenter, 1971, §24-39}

\subsubsection{Dommen}
Ifølge Prenter afhænger dommen af Kristi genkomst på baggrund af, at Menneskesønnen, ud fra bibelske
udsagn, forventes at komme igen som dommer. Prenter opfatter spørgsmålet om menneskets skæbne hinsides
den fysiske død at være skjult i den yderste dom. Grundlaget for dommen er ifølge Prenter forholdet til
Jesus Kristus. Derved kommer spørgsmål om tro til at spille en stor rolle. Det vil sige, dommen tager udgangspunkt i den enkeltes tro på Jesus eller manglen på det samme.\footnote{Prenter, 1971, s. 604} Dens funktion er at åbenbare menneskets skjulte liv og at anerkende Kristi herlighed. På den måde åbenbares der i dommen, hvor skillelinjen mellem de fortabte og de frelste går.\footnote{Prenter, 1971, s. 602}

Det, at dommen er skjult, er for Prenters en vigtig pointe i forhold til tanken om prædestination. Prenter opfatter prædestinationen som en gåde. Dette således, at frelsen kun kan søges via troen på Jesus Kristus, så længe gåden forbliver. Derved kan dommen holdes ved sit rette grundlag, i den forstand, at dommen er Guds anliggende.\footnote{Prenter, 1971, s. 605}

Prenters lære om dom omfatter også, at dommens udgang er dobbelt. Den ene udgang fører til frelse og den anden fører til fortabelse. Dette hænger tæt sammen med Prenter opfattelse af, at troen er grundlag for dommen. For, hvis frelsen beror på troen på Kristus, så må fortabelsen bero på at miste Kristus.\footnote{Prenter, 1971, s. 617} Prenter gør i samme omfang opmærksom på, at fortabelsen ligesom prædestinationen skal forstås som en gåde. Dette forstået således, at gåden i sig selv er en advarsel om, at mennesket skal holde sig til troen og stole på ordet.\footnote{Prenter, 1971, s. 619, for uddybelse se Matt. 8:12, 18:8-9, 23:33, 25:41 }
\chapter{Apokatastasis}
Apokatastasis er læren om altings genoprettelse. Dette betyder en universel frelse, der omfatter, at alt vender tilbage til sin oprindelige tilstand. Læren har fokus på Kristus som har givet sit liv som løsesum for alle. Dertil indebærer læren et centralt håb om, at Gud vil sin skabelses finitte frelse. Apokatastasislæren indgår i Jürgen Moltmanns og Gregor af Nyssas opfattelse af alles frelse. 
\chapter{Skærsilden}
\textit{Den kateketiske tale} og \textit{Om sjælen og opstandelsen} er dogmatiske skrifter af Gregor af Nyssa. I disse skrifter gør Gregor rede for sin forestilling om skærsilden. Tanken med skærsilden omfatter, at den menneskelige sjæl skal igennem en rensende ild for at opnå endelig frelse. Det essentielle ved idéen om renselsen er, at noget skal tilintetgøres for at nyskabelsen kan vokse frem i det gamles sted.

Det centrale for Gregors forestilling er, at han opfatter mennesket som værende født med en frihed men også med en indre længsel. Denne længsel er en bevægelse i sjælen mod Gud. Dette forstået således, at mennesket, på grund af sjælen, kender til Guds eksistens.1 Dog er dette ikke ensbetydende med at vide, hvad Gud er. For for mennesket er han ubegribelig. Derfor bliver menneskets bevægelse mod Gud en stadig udgriben efter Gud. Dette skal forstås som, at mennesket, ved at se på verdenen, i sig selv danner et billede af, hvad Gud måtte være. Men billedet kommer altid at være en håndsudrækkelse mod Gud, hvor man aldrig helt når op til Gud.

Essentielt for Gregors opfattelse er, legemets fornuft, at sjælen er skabt og det, der bistår legemet i dets vitalitet.2 Ydermere forstår Gregor evnen til at tænke og overveje som et guddommeligt karaktertræk. Det vil sige, at sjælen er beslægtet med Gud idet, at den bærer efterligninger til det guddommelige i form af en genspejling.3 Det er nemlig på grund af det guddommelige aftryk i sjælen, at mennesket har en eksistentiel længsel mod det guddommelige.4 Sjælen har ifølge Gregor taget bo i legemet, men på grund af dens slægtskab med Gud, så er den forskellig fra legemets faste stof.

\section{Opstandelsen i lyset af døden}
Gregors opfattelse af opstandelsen skal ses i lyset af døden. Ifølge ham vil alle mennesker stå op fra de døde -- nogle til dom og andre til frelse. Der er derfor ingen grund til at frygte døden, hvis man kender til dens betydning. For Gregor er døden nærmest en gave, hvis den rettelig forstås i lyset af opstandelsen. Med reference til 1. Mos. 2:21 redegør Gregor for, at dengang Gud iklædte de første mennesker i skind, gav han dem osgå egenskaben til at dø. Dette forstået således, at klæderne eller rettere dødeligheden omslutter mennesket udefra, men dets indre guddommelige natur er trods dødelighedens kåbe intakt.5 Dødens funktion markerer i denne forbindelse en overgivelse af mennesket til udødeligheden, idet Gud ”gennem opstandelsen [former] en krukke på ny, der er renset for det onde ved hjælp af opstandelsen.”6 Gregor ser derfor opstandelsen som et bindeled mellem legeme og sjæl, og kalder denne forening 'mysteriet i Guds frelsesplan'. Foreningen mellem legeme og sjæl symboliserer mødestedet for liv og død, hvor Gud og mennesket kan forenes med hinanden.7 Døden bliver på en måde et redskab for Gud at ophæve synden. Således forstået, at døden i sig selv er negativ indtil Kristus forener død og liv, og derved muliggør den treenige økonomis forening.

\section{Den rensende ild}
Skærsilden har en vigtig funktion for foreningen mellem menneske og Gud. Mennesket er som før nævnt en sammensat enhed. Ifølge Gregor er det når sjælen kommer i kontakt med det legemlige liv igennem sanserne, at sjælen kan tilsmudses. Dette er tilfældet, hvis fornuften mister magten over lidenskaberne, hvilket resulterer i, at mennesket synker fra det fornuftige og guddommelige til det ufornuftige og dåreagtige.8 Lidenskaber sætter sig i så fald fast på sjælen og forårsager, at den gøres mere stoflig.9 Dette resulterer i, at sjælen har brug for en renselse. Denne renselse medfører en erfaring af store smerter, ”når den guddommelige kraft af kærlighed til mennesket trækker det, der er dens eget, ud af de ufornuftige og stoflige ruiner.”10 Tilbage står det ufuldkomne i mennesket, skidtet, som ifølge Gregor skal renses, fortæres og tilintetgøres ved ild. Gregor forstår skærsilden som Guds kærlighed til mennesket, fordi denne muliggør alles frelse. Skærsilden som dom er derfor ikke Guds måde at påføre lidelse, men at skille godt fra det onde.

Renselse og frelse er ifølge Gregor Guds eneste mål. Det vil sige, at alle, når menneskets naturs fylde er blevet fuldkommen, vil få del i det gode, som findes i Gud.11 Videre anfører Gregor, at opstandelsen har den betydning, at menneskets natur vender tilbage til dens oprindelige tilstand. Denne opfattelse tager udgangspunkt i apokatastasislæren. Ifølge Gregor genvinder alle den oprindelige fuldkommenhed, og denne genoprettelsen opfatter han som en helbredelse. Helbredelsen omfatter, at det, der har blandet sig med sjælen, renses og aflægges.12 Idet sjælen besidder bevidstheden, er det kun sjælen, der kan erfare døden, når legemet dør. Straffens form er ifølge Gregors transformeret til kun at omfatte en renselse, igennem hvilken Gud drager mennesket til sig og omformer dets sjæl til at omfatte sjælens oprindelige renhed.
\chapter{Universalisme}
I bøgerne The Coming of God (CG) og Theology of Hope (TH) redegør Moltmann for sin opfattelse af den kristne eskatologi. Indenfor den teologiske retning kaldet apokatastasis repræsenterer han tanken om universalisme, som omfatter tanken om håbet på en universel frelse og restaurationen af alle ting. Spørgsmålet om alles frelse er et eskatologisk spørgsmål, som, ifølge Moltmann, teologisk set kun kan besvares kristologisk.1 Dette bliver også klart igennem hans eskatologiske tænkning, hvor Kristus på korset opfattes som svaret på det eskatologiske håb.

Fænomenet håb står centralt i Moltmanns forestilling om universalisme. Under betegnelsen 'personlig eskatologi' redegør han for individets personlige håb på evigt liv. Historisk eskatologi spiller derimod en rolle for individet i forhold til det universelle håb. Dette ud fra opfattelsen af, at mennesket er en integreret del af en større sammenhæng.2 Men ifølge Moltmann hænger den personlige og den historiske eskatologi tæt sammen, for det universelle håb skal omfatte håbet om evigt liv for hele skabelsens univers.  

\section{Personlig eskatologi}
Ifølge Moltmann omhandler personlig eskatologi det enkeltes menneskes tro og håb på det evige liv. Det
vil sige, at omdrejningspunktet for håbet er den eskatologiske begivenhed, hvori menneskets blik er rettet mod tidspunktet for Kristi genkomst. Dette er ifølge Moltmann indledningen på fornyelsen af alle ting. Videre synes han, at al teologi skal ses i et eskatologisk perspektiv. For alle de eskatologiske begivenheder -- den universelle herlighed, den sidste dom, opstandelsen af de døde og nyskabelsen af alle ting -- har en lærende funktion for livet. 3 Uden eskatologien som del af nutiden har kristendommens lære ingen frugtbar eksistens, idet den berøves den logiske nødvendighed af korset og opstandelsen, Kristi ophøjelse og suverænitet.4 Sammenhængen i dette er, at eskatologiens håb skal være konstituerende i livet her, hvis håbet, som mennesket holder sig til, skal have en reel betydning, når mennesket står ansigt til ansigt med døden. Dette er, hvad Moltmann forstår ved Wolf Biermanns digt ”we have to get life into life.”5

Moltmann indleder afsnittet \textit{Personlig Eskatologi} (CG) med menneskets ambivalente forhold til liv og død, ved at fremstille Epikurs opfattelse af forholdet mellem liv og død. Ifølge Epikur kan liv og død kun eksistere adskilt fra hinanden. Hvorimod Moltmann gør opmærksom på, at de to fænomener er tæt forbundet i et gensidigt afhængighedsforhold og eksisterer altid på samme tid. Moltmann opfatter livet og døden som fundamentale oplevelser og ikke blot biologiske fænomener,6 som indadtil forudsætter, hvordan mennesket opfatter og reflekterer over livet som helhed. Derfor må en eskatologi, der vedrører det personlige, inddrage hele menneskets levegrundlag for at bevare dens autenticitet i forhold til håbet. Ifølge Moltmann er håbet forudsætningen for, at mennesket på ingen måde kan affinde sig døden som den blotte død. I kraft af håbet forholder mennesket sig utrøsteligt i forventningen af et liv, der overvinder døden,7 og kristendommen må gå denne forventning i møde. Ifølge Moltmann handler kristendom helt og holdent om eskatologi på grund af dets håbs fremadrettede retning mod en forsoning og forløsning. Herfor er Gud omdrejningspunktet og igennem Kristus har Gud åbenbaret sin fremtid, som er en forsoning uden grænser.8 Ifølge Moltmann er oplevelsen af Kristi død og opstandelse formgivende for den kristne tro. Begivenheden på korset og opstandelsen kommer til at fungere som indgangen til det evige liv.9 Ydermere kommer samme begivenhed at stå som grundlag for menneskets håb på en forsoning uden grænser.10

Det kristne håb er ifølge Moltmann universalistisk i sin karakter, fordi det håber på et liv, der overvinder døden. Det vil sige, at kristendommen håber på en livgivende Gud, der overvinder døden sådan som 1. Kor. 15:54 giver udtryk for.11  Transformationen har i denne forbindelse betydning for tanken om dødens opstandelse. Ud fra 1. Kor. 15:52 og Phil. 3:21 forstår Moltmann transformation som en helbredelse, forsoning og fuldbyrdelse. Transformationen er mulig i den forstand, at Guds ånd indenfor kristendommen tænkes at være livgivende og dertil den virkende kraft i opstandelsen.12 Ånden muliggør derved en konfiguration af livet som helhed, der medfører, at mennesket i kraft af Kristi opstandelse kan opleve det fysiske liv som både dødeligt og udødeligt på samme tid.13 Det vil sige, at opstandelsen betyder en indoptagelse af det fysiske liv ind i det evige liv, som for mennesket betyder, at fremtiden er medtænkt i nutiden. Denne fremstilling af tiderne er fra Moltmanns side et forsøg på at sætte tiderne i relation til hinanden i forhold til, hvordan mennesket kan tænke kontinuitet og evigt liv ind i diskontinuiteten, som døden i nuværende liv forudsætter.

Moltmann forstår transformation således, at Gud præserverer menneskets selv og transformerer det for at perfektionerer det. Dette på den måde, at mødet med døden fungerer som en tillære eller en ny orientering, hvor mennesket forstår noget nyt i forhold til sig selv og i forhold til Gud.14 Ifølge Moltmann er livshistorien og –erindringen i den forstand vigtige at få med ind i det ny transformerede liv. Ligesom Kristus igennem transformationen beholdt sine identifikationsmærkater, således vil mennesket også igennem transformationen bevare sine kendetegn.15 Moltmanns pointe er, at mennesket igennem døden skal opnå sjælero ved det levede liv. Dette på den måde, at mennesket, i mødet med Gud, møder sig selv. Døden er i den forstand ikke slutningen, men en proces. Dette må forstås i lyset af Moltmanns opfattelse af, at alle mennesket har levet halve og ufuldendte liv, og at Gud igennem apokatastasis giver mennesket muligheden for at udleve dit fulde potentiale.16 

\section{Historisk Eskatologi} 
Historisk eskatologi omfatter et universelt håb om forløsning for ethvert individ samt for hele universet. I denne forbindelse gør Moltmann opmærksom på, at historisk eskatologi afhænger af den kosmiske eskatologi, nyskabelsen af verden. Ydermere er der ifølge Moltmann ingen personlig eskatologi uden en transformation af de kosmiske vilkår. Det vil sige, at transformationen eller forløsningen ikke kun er indsnævret til at omfatte en lille grubbe udvalgte, men at forløsningen omfatter alt liv i universet som helhed.17 Dommedag opfattes af Moltmann som den universelle åbenbaringen af Kristus og fuldbyrdelsen af hans forløsende handling på korset. Ud fra dette gør Moltmann opmærksom på, at det sande grundlag for kristendommens håb på universel frelse er en teologi, hvis omdrejningspunkt er korset. For at forklare fokusset på korset beskriver Moltmann blandt andet Luthers og Hans Urs von Balthasars forestillinger om betydningen af Kristi nedfart til dødsriget. Luther og von Balthasar opfatter Kristi nedfart til helvede som en eksistentiel erfaring. Ifølge Luther omfatter erfaringen, at Guds vrede og forbandelse er over synd og gudløs væren, således at Kristus led helvede på korset for at forsone denne verden med Gud.18 Ifølge Luther blev Kristus synd for at mennesket skulle opnå forsoning. Von Balthasars opfatter Kristi nedfart påskelørdag som alles frelse. I nedfarten oplevede Kristus den absolutte gudsforladthed, således at menneskeheden kunne være omfattet af solidariteten. Det vil sige, at selv i helvedes dybder finder mennesket Kristus som sin frelser og derigennem også Gud.19 Moltmanns pointe i dette er, at der for mennesket er håb at finde i en teologi med fokus på korset. Budkabet i en sådan en teologi er, at ”Christ gave him self up for lost in order to seek all who are lost, and bring them home.”20 En teologi med dette fokus kan i Moltmanns optik kun resultere i restaurationen af alle ting. For det, som Kristus opnår i lidelse, død og opstandelse, vil manifestere sig i herlighed igennem hans parousia.21 Ud fra dette konkluderer Moltmann, at dommedag ikke har med frygt og bæven at gøre, men dommedag skal hellere opfattes som en stor og glædelig sejer. Moltmann gør opmærksom på, at der er to sider af den eskatologiske doktrin om alle tings restauration. For det første omfatter den Guds dom, at alle ting retfærdiggøres, og for det andet så vækkes Guds rige til nyt liv i den restaurerede skabelse.22
\chapter{Annihilationsteorien}
Niels Henrik Gregersen søger i sin artikel Karl Barth og annihilationsteorien at redegøre for Barts position mellem apokatastasis læren og annihilationsteorien.
I den Kirchliche Dogmatik af Barth fremstiller han en nyfortolkning af udvælgelses læren, hvor Gud ”i Kristus […] evigt har udvalgt sig mennesket som sin partner og evigt har valgt sig selv til at bære syndens forbandelse.”1 Barths opgave går ud på at fremstille prædestinationen med nåden som udgangspunkt. Det vil sige, at nåden bliver af Barth præsentæret på dens egen vilkår i modsætning til Augustins traditionelle lære om nåden, som i større grad indgyder skræk end barmhjertighed. I Barths optik er baggrunden for Guds udvælgelse baseret på et frit nådesvalg og forbliver af selvsamme grund en uselvfølgelighed.2 Barth er tilhænger af tanken om en dobbelt udgang og afviser apokatastasis læren. Dog forekommer Barts synspunkter ifølge Gregersen at modstride hinanden idet den kristologiske struktur i Barths udvælgelseslære er universalistisk formuleret. I den forbindelse fremsætter Gregersen muligheden for at kunne forklare sammenhængen mellem den dobbelte udgang og den frelsesuniversalisme, der gør sig gældende i udvælgelseslæren, ud fra annihilationsteorien.

Annihilation omfatter tanken om den evige død på den måde, at menneskets eksistens bliver annihileret, tilintetgjort. Ifølge denne teori bliver pointen i Guds nådesvalg, at synden skal tilintetgøres og synderen skal leve og frelses. Derigennem forandres karakteren af dommens dobbelte udgang fra den klassiske Augustinske forståelse, hvor udgangspunktet for dommen omfatte to grupper opdelt i frelste og fortabte, til at omfatte to sider af samme menneske, den der skal frelses og den der skal forgå. Hovedpointen i dette er for Barth, at alle mennesker er omfattet af Guds nåde, men dertil er der noget i alle mennesker der skal forgå. Det vil sige at dommen går midt ned igennem hvert eneste menneske.3 Dommen bliver derved Guds redningsaktion for mennesket. Ifølge Gregersen overvinder Guds nåde over Guds dom. Det vil sige, at alle mennesket i Barths optik står på Guds frelsesvej. Han afviser i den forbindelse tanken om massa perditionis med opfattelsen, at menneskets nej til Gud er af Gud dødsdømt.4 Dette forstået på den måde, at Guds ja til mennesket altid vil overtrumfe menneskets nej, fordi nejet bliver annihileret. Ifølge Gregersen er der for Barth en tæt og afgørende forbindelse mellem Guds dom og Guds retfærdighed. Forbindelsen går ud på, at Gud altid vil møde mennesket på dets egne præmisser. Gergersen forklarer det således: ”Gud retfærdiggør den kristne som hedning, gud-løs.”5 

I Barths opfattelse af Guds kærlighed gør Gregersen opmærksom på, at han forbinder kærligheden med tanken en rensende ild. Ifølge Gregersen har denne ild til formål at fortære menneskets egetliv til fordel for at lade Kristus vokse frem. Derved kommer Guds tilintetgørelse af mennesket at stå i frelsens tjeneste.6 Ud fra dette Gregersen beskriver Gud udvælgelse som værende menneskets dødsrute der transformeres til en livslinje. Denne form for spil mellem modsætninger er kendetegnende for Barts teologi, mere kendt under betegnelsen dialektisk teologi. Dialektikken kommer bl.a. til syne i Barths tanke om, at alle må dø for at tage del i Guds nye liv.7’Dødsruten som livslinje’ forekommer at være et centralt modsætningspar der beskriver Barths tanke om, at det nye menneske skal opstå ud af det gamle menneske, og derved giver Guds dom synden nådesstødet.8 Det vil sige, at annihilationen af synden muliggør Guds tilblivelse som alt i alle. 

I en afsluttende perspektivering gør Gregersen opmærksom på, at apokatastasis ikke kan inkorporeres i kirkelæren. Dette fordi begrænsningen i menneskets erkendelsesevne og respekten for Guds frihed må fastholdes.9 Tilsvarende dette forhold skriver Gregersen, at kirken må regne alle mennesker som del af Guds udvælgelse. Således bliver spørgsmålet om apokatastasis i Gregersens optik Guds mulighed og menneskets eneste grundlag for håb. 
\chapter{Diskussion}
Gregersens artikel Karl Barth og annihilationsteorien indledes således: ”netop i evighedens lys skal der aflægges regnskab for hvert øjeblik. Netop over for evigheden bliver tiden dyrebar.”1 Ydermere er evigheden konstituerende og limitativt begreb for tiden. Som konstituerende for tiden betyder evigheden, at tiden til evigt forbliver en ’uindhentelig fremtid’ og i kristen forstand kommer mennesket derved til enhver tid at stå overfor Guds dom.2 Samtidig er evigheden et limitativt begreb, fordi åbenbarelsen af dommens dag forbliver i den evige fremtid.3 I den forstand står mennesket imellem en Guds dom og spørgsmålet om der overhoved er en Gud eller en dom. Mennesket er fanget midt imellem, fordi døden tillader intet andet. Døden fordrer mennesket til forestillinger om, hvad døden har som følge. Disse forestillinger forholder sig til spørgsmålet om Gud, nåde eller frelse. I forhold til teologerne Augustin, Prenter, Moltmann, Gregor, Barth og Gregersen forekommer sammenhængen i ens forestilling om efterlivet at afhænge af menneskets egen forståelse af livet her. Forstået på den måde, at livet bliver konstituerende for efterlivet. Dommen er nødvendig for den røde tråd, der forbinder livet og dets kontinuitet i dødens opstandelse.

Ifølge Gregor forestilling forekommer Guds vilje til frelse at være en selvfølgelighed. Dertil forudsætter slægtsforholdene mellem mennesket og Gud, at mennesket nødvendigvis drages i retning af Gud. Gregor har en tillid til selvfølgeligheden i Guds grænseløse kærlighed, der resulterer i alles frelse. Denne tiltro til Guds grænseløse kærlighed tillægger Moltmann også sin lære om det universale håb på frelse. Hans forestillingen forholder sig til en genoprettes der omfatter alle. Baggrunden herfor er håbet på en universel retfærdighed. I en refleksion over uretten livet ofte møder mennesket med, fremsætter Moltmann et håb på, at døden er en rehabiliteringsproces for mennesket. Det transcendente håb kommer for Moltmann at være forudsætningen for menneskets fulde oplevelse af dets eksistens. Det vil sige, at opstandelses håb sejrer over døden midt i livet og efterlader ikke mennesket i en cliff-hanger, som Moltmann forklarer således: ”I shall live wholly here, and die wholly, and rise wholly there.”4 Ifølge Moltmann er det ikke menneskets tro, der afgør dets frelse, men derimod skaber frelsen troen.5 Dette udtryk kan også gøre sig gældende for Gregors opfattelse.

På den anden side fordre døden, at man som menneske tænker over ens livsførelse. Sætninger som ”Lev som om du skulle dø i morgen”, ”Du ved aldrig, hvornår du har levet din sidste dag” og ”Memento mori” er udtryk for just dette. Tilsvarende disse udsagn synes Augustin og Prenter at være enige om, at livsførelse har noget at sige for, hvilket efterliv en idømmes. Augustin lægger vægt på, at ret livsførelse der søges igennem tro er kriteriet for belønning af evigt liv i himmelen. Ifølge Prenter forholder spørgsmålet om himmelen sig til en fornyelse der har retning mod fuldendelsen i herlighedens evighed. Fornyelsen ifølge Prenter er en livsførelse i fællesskab med kirkens, da kirkens menighed er Guds nye pagts gudsfolk. Grundlaget for ens frelse opfattes af Prenter at være forholdet til Kristus. Derimod forekommer det for dem, at et menneske, der ikke forholder sig til Kristus og Gud i sin livsførelse kommer at idømmes helvede som straf herfor.

Barths forestilling af menneskets livsmuligheder foreholder sig tilsvarende Augustin og Prenter til en dobbelt udgang. Modsat Augustin og Prenter foregår der ikke en opdeling af mennesker, men en opdeling i mennesket. Dommen går direkte ned igennem mennesket, hvor synden i mennesket skal tilintetgøres, men mennesket som synder skal reddes. I den forstand repræsenterer Barth sammen med Gregor og Moltmann frelsesuniversalisme.

Teologernes anstrengelser for at skabe en tese vedrørende Gud forbliver aldrig mere end just det, en tese. Så hvorfor gøre sig disse anstrengelser? En anden mulighed ville være at forholde sig til dette liv som om intet kom bagefter. Epikur mente, at ”mens vi er, er døden ikke. Mens døden er, er vi ikke.” Det vil sige, at der ingen transcendent verden findes udenfor vor erfaringskategorier. Mennesket oplever nu og engang holder det op med at opleve. Det er dog ikke således mennesket indretter sit verdens billede. Mennesket søger konstant at skabe mening ud af fortiden for at forholde sig til nutiden og fremtiden. Det vil sige, at mennesket i sine tanker krydser dødens grænse både baglens og forlæns. Ud af dette kan man sige, at forudsætningen for domforestillingerne er døden og det, at mennesket ikke kan acceptere den som blot en død. Måske giver forestillingerne livet lidt mere nuance og rumlighed.  
\chapter{Konklusion}
Konklusionen er, at der ingen konklusion er for, hvilken domsforestilling kommer tættest på sandheden. I indledningen til sin artikel skriver Gregersen, at ”netop over for evigheden bliver tiden dyrebar.”1 Er det ikke netop det, der er den højeste værdi, at forestillingerne blot er forestillinger? Mennesket er derved frit stillet til at fortolke livet og døden. Jeg forstår den guddommelig stilhed som en respekt menneskets ret til egen forståelse af livets mening. Cliff-hanger er i den forstand en forstyrrelse, men her kommer forestillingerne ind i billedet som et tilbud om mulig fortolkning af meningen med livet. I CG giver Moltmann udtryk for at ”life is not an experiment.” Dette synspunkt er set ud fra håbet om alles frelse. Men ud fra mine menneskelige erfaringer ville jeg vove at sige, at livet netop er et eksperiment – et dødsseriøst eksperiment. Håbet er, at eksperimentet giver mening nok til, at man, når dødens er nær, kan finde ro med at passere denne definitive dørtærskel. I tesen satte jeg spørgsmål med meningen i forestillingerne. Konklusion er, at det for nogle mennesker giver en sans for dybere mening i livet at håbe på noget. Teologen Thomas Talbott udtrykt det således, at når alt kommer til alt, er det op til Gud, hvordan han på bedste vis vil åbenbarer sig for os.2
\chapter{Litteraturliste}
\begin{itemize}
\item Augustin: Kristendomsundervisning for Begyndere. Tro, Håb og Kærlighed. En Håndbog; Augustins Småskrifter om Troen. Bind 2, oversættelse: Torben Damsholdt. (København: Forlaget Anis, 2005.) s. 7 – 133 (126 sider)
\item Walls, Jerry L.: ”Heaven”; The Oxford Handbook of Eschatology, red. Jerry L. Walls (Oxford University Press, 2008) s. 399 – 412 (13 sider)
\item Kvanvig, Jonathan L.: ”Hell”; The Oxford Handbook of Eschatology, red. Jerry L. Walls (Oxford University Press, 2008) s. 413 – 426 (13 sider)
\item Griffiths, Paul J.:  ”Purgatory”; The Oxford Handbook of Eschatology, red. Jerry L. Walls (Oxford University Press, 2008) ss. 427 – 445 (17 sider)
\item Talbott, Thomas: ”Universalism”; The Oxford Handbook of Eschatology, red. Jerry L. Walls (Oxford University Press, 2008) s. 446 – 461 (15 sider)
\item Pinnock, Clark H.: ”Annihilationism”; The Oxford Handbook of Eschatology, red. Jerry L. Walls (Oxford University Press, 2008) s. 462 – 475 (13 sider)
\item Gregersen, Niels Henrik: ”Karl Barth og annihilationsteorien”; Dansk Teologisk Tidsskrift 63 årg., 2000. s. 134 – 155 (21 sider)
\item Moltmann, Jürgen: “Is there life after death?”; The End of the World and the Ends of God. Science and Theology on Eschatology, red. John Polkinghorne and Michael Welker (Trinity Press International, 2000) s. 238 – 255 (17 sider)
\item Moltmann, Jürgen: The Coming of  God, Christian Eschatology; oversættelse: Margret Kohl (Minneapolis: Fortress Press, 1996) s. 58 – 110 (52 sider)
\item Moltmann, Jürgen: Theology of Hope; oversættelse: James W. Leitch (SCM Press Ltd.: 1967) 5. Oplag. s. 15 – 19 (4 sider) 
\item Hunsinger, George: “Hellfire and Damnation: Four Ancient and Modern Views”; Scottish journal of Theology Ltd: 1998, Volume 51, Issue 04, November, 1998. s. 406 – 434 (28 sider)
\item Jacobsen, Anders-Christian: “Augustin om menneskets opstandelse”; Dansk Teologisk Tidsskrift 65 årg., 2002. s. 255 – 271 (17 sider)
\item Prenter, Regin: Skabelse og Genløsning; Dogmatik (København: G.E.C. Gads forlag, 1971) s. 216 – 624 (408 sider) 
\item Nyssa, Gregor af: ”Antikken og Kristendom 5”; oversættelse og indledning: Jørgen Ledet Christiansen, red. Niels Arne Pedersen (København: Forlaget Anis, 2007) s. 57 – 234 (177 sider)  
\item Augustin: Om Guds Stad.  Oversættelse med indledning: Bent Dalsgaard Larsen (Aarhus Universitets forlag, 2002) s. 99 – 116 (17 sider) og s. 803 – 943 (140 sider)
\item Gregersen, Niels Henrik: Ind i Gud: Om at tænke opstandelse og evigt liv i dag. (Uudgivet tekst: 2015) s. 1 - 26 (26 sider)
\end{itemize}

\textbf{\underline{\large I ALT ca. 1104 sider}}
\end{document}