\chapter{Teori og metode}
Afsnit \ref{chp:esk} fungerer som teoretisk baggrund for de forestillinger teologerne nævnt i afsnit \ref{chp:kilder} har om kristendommens lære om eskatologi og hvad denne lære omfatter af begivenheder. Begreberne dobbelt udgang, apokatastasis og annihilation er centrale for forståelsen af pågældende teologers teser om den yderste dom. Desuden specificerer nævnte begreber, hvordan disse teologer tænker Guds nåde og frelse at komme til udtryk.

Selve opgaven forholder sig systematisk analytisk til forestillingerne om dom i deres varierende former og fremstiller dem hver for sig. Hver enkelt af fremstillingerne er afgrænset i sin redegørelse og forholder sig til tanken om den yderste dom. Indledningsvis er der en forklarende fremstilling af kristen eskatologi baseret på The Oxford Handbook of Eschatology. Afsnit \ref{chp:disk} indeholder en komparativ analyse af forestillingerne. Den komparative analyse kommenterer enkeltstående synspunkter fra ovenfornævnte forestillingerne og sætter dem i forhold synspunkter, som \textit{The Oxford Handbook of Eschatology} debatterer. Analysen opstiller sammenhængen af det fokus forestillingerne har på liv, død og efterliv. Den komparative analyse suppleres af en personlig refleksion over forestillingernes fokus sat i forhold til mennesket.