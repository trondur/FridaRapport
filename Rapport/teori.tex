\chapter{Teori og metode}
Forestilling som teologerne nævnt i afsnit \ref{chp:kilder} har om de eskatologiske begivenheder fungerer som baggrund for teorien om kristendommens eskatologi, og hvad den omfatter. Begreberne dobbelt udgang, apokatastasis og annihilation er centrale for forståelsen af pågældende teologers teser om, hvad eskatologien omfatter. Desuden specificerer nævnte begreber, hvordan guds nåde og frelse kommer til udtryk.

Selve opgaven forholder sig systematisk analytisk til forestillingerne om dom i deres varierende former og fremstiller dem hver for sig. Hver enkelt af fremstillingerne er afgrænset i sin redegørelse og forholder sig til tanken om den yderste dom. Indledningsvis er der en forklarende fremstilling af kristen eskatologi baseret på The Oxford Handbook of Eschatology. Afsnit ”REFREFREF”Diskussion indeholder en komparativ analyse af forestillingerne. Analysen opstiller sammenhængen af det fokus forestillingerne har på liv, død og efterliv. Denne komparative analyse suppleres af en personlig refleksion over forestillingernes fokus sat i forhold til mennesket.