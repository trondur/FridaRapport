\chapter{Annihilationsteorien}
Annihilationsteorien omfatter tanken om straffen som en tilintetgørelse. Det vil sige, at Guds straf for ondsindede mennesker er en tilintetgørelse af deres eksistens. Annihilationsteorien, der bliver fremstillet i nedenforstående redegørelse af Karl Barths udvælgelseslære, afviger på nogle punkter fra den klassiske annihilationsteori. Denne udgave af annihilationsteorien omfatter, at dommen går ned igennem mennesket. Det vil sige, at dommen ikke omfatter en opdeling mellem dem der frelses, og dem der til intetgøres. Men at dommen rammer to sider af mennesket, den del af mennesket, der frelses, og den, der tilintetgøres. Ifølge Niels Henrik Gregersen er denne reviderede annihilationstanke at finde i Karl Barths udvælgelseslære.\footnote{Gregersen, 2000, s. 155}   

\section{Annihilation med alles frelse}
Niels Henrik Gregersen søger i sin artikel \textit{Karl Barth og annihilationsteorien} at redegøre for Barts position mellem apokatastasis læren og annihilationsteorien. I \textit{Kirchliche Dogmatik} af Barth fremstiller Barth en nyfortolkning af udvælgelseslæren, hvor Gud ”i Kristus […] evigt har udvalgt sig mennesket som sin partner og evigt har valgt sig selv til at bære syndens forbandelse.”1 Barths opgave går ud på at fremstille prædestinationen med nåden som udgangspunkt. Det vil sige, at nåden bliver af Barth på dens egne vilkår i modsætning til Augustins traditionelle lære om nåden, som i større grad indgyder skræk end barmhjertighed. I Barths optik er baggrunden for Guds udvælgelse baseret på et frit nådesvalg og forbliver af selvsamme grund en uselvfølgelighed.2 Barth er tilhænger af tanken om en dobbelt udgang og afviser apokatastasis læren. Dog forekommer Barts synspunkter ifølge Gregersen at modstride hinanden, idet den kristologiske struktur i Barths udvælgelseslære er universalistisk formuleret. I den forbindelse fremsætter Gregersen muligheden for at kunne forklare sammenhængen mellem den dobbelte udgang og den frelsesuniversalisme, der gør sig gældende i udvælgelseslæren, ud fra annihilationsteorien.

Annihilation omfatter tanken om den evige død på den måde, at menneskets eksistens bliver annihileret, tilintetgjort. Ifølge denne teori bliver pointen i Guds nådesvalg, at synden skal tilintetgøres og synderen skal leve og frelses. Derigennem forandres karakteren af dommens dobbelte udgang fra den klassiske Augustinske forståelse, hvor udgangspunktet for dommen omfatte to grupper opdelt i frelste og fortabte. I stedet omfatter den to sider af det samme menneske. Den side, der skal frelses, og den, der skal forgå. Hovedpointen i dette er for Barth, at alle mennesker er omfattet af Guds nåde, men dertil er der noget i alle mennesker, der skal forgå. Det vil sige, at dommen går midt ned igennem hvert eneste menneske.3 Dommen bliver derved Guds redningsaktion for mennesket. Ifølge Gregersen vinder Guds nåde over Guds dom. Det vil sige, at alle mennesket i Barths optik står på Guds frelsesvej. Han afviser i denne forbindelse tanken om 'massa perditionis' med opfattelsen, at menneskets nej til Gud er dødsdømt af Gud.4 Dette forstået på den måde, at Guds ja til mennesket altid vil overtrumfe menneskets nej, for nejet bliver annihileret. Ifølge Gregersen er der for Barth en tæt og afgørende forbindelse mellem Guds dom og Guds retfærdighed. Forbindelsen er, at Gud altid vil møde mennesket på dets egne præmisser. Gergersen forklarer det således: ”Gud retfærdiggør den kristne som hedning, gud-løs.”5 

I Barths opfattelse af Guds kærlighed gør Gregersen opmærksom på, at han forbinder kærligheden med tanken en rensende ild. Ifølge Gregersen har denne ild til formål at fortære menneskets egetliv til fordel for at lade Kristus vokse frem. Derved kommer Guds tilintetgørelse af mennesket at stå i frelsens tjeneste.6 Ud fra dette beskriver Gregersen Guds udvælgelse som værende menneskets dødsrute, der transformeres til en livslinje. Denne form for spil mellem modsætninger er kendetegnende for Barts teologi -- mere kendt under betegnelsen dialektisk teologi. Dialektikken kommer blandt andet til syne i Barths tanke om, at alle må dø for at tage del i Guds nye liv.7 'Dødsruten som livslinje' forekommer at være et centralt modsætningspar, der beskriver Barths tanke om, at det nye menneske skal opstå ud af det gamle menneske. Derved giver Guds dom synden nådesstødet.8 Det vil sige, at annihilationen af synden muliggør Guds tilblivelse som alt i alle. 

I en afsluttende perspektivering skriver Gregersen, at kirken må regne alle mennesker som del af Guds udvælgelse. Således bliver spørgsmålet om apokatastasis i Gregersens optik Guds mulighed og menneskets eneste grundlag for håb. 