\chapter{Kilder}
\label{chp:kilder}
Kilder til forståelse af Augustins tanke om en prædestineret dobbelt udgang er baseret på hans egne værker Enchiridion (Ench.) og De civitate Dei (De civ. Dei). Yderlig viden omkring Augustins teologi er hentet fra Anders-Christian Jacobsens artikel Augustin om menneskets opstandelse og Georg Hunsingers artikel Hellfire and Damnation: Four Ancient and Modern views. Kildematerialet til forståelse af Regin Prenters opfattelse af prædestineret dobbelt udgang er baseret på Prenters eget dogmatiske værk Skabelse og Genløsning. Til fremstilling af Jürgen Moltmanns opfattelse af frelsesuniversalisme har hans eskatologiske værker Theology of Hope og The coming of God fungeret som kildemateriale. Kildemateriale til forståelse af Gregor af Nyssas tanker om alles frelse er hentet fra bogen Gregor af Nyssa. Bogen indeholder oversættelser af tre af hans værker Markinas Liv, Om sjælen og Opstandelse og Den Kateketiske Tale. Hele samlingen har fungeret som baggrunds viden om Gregor. Artiklen Barth og annihilationsteorien skrevet af Niels Henrik Gregersen er eneste kildegrundlag til fremstilling af Barths tanke om dobbelt udgang og hans udvælgelseslære. The Oxford Handbook of Eschatology fungerer som kilde til afsnit \ref{chp:esk} Eskatologi, som er en forklarende oversigt over emner, der har med forestillinger om himmel, helvede, frelsesuniversalisme og annihilation at gøre.