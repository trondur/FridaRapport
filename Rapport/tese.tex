\chapter{Tese}
De eskatologiske begivenheder ligger udenfor menneskets erfaringsevne, men det holder ikke mennesket fra at tænke over det. Spørgsmålet er, hvilken mening det giver at tænke ud over de grænser, hvor vores sanser ikke rækker? Forestillingerne forbliver ikke andet end forestillinger. Erfaringsgrænsen møder dog mennesket med en guddommelig stilhed,1 som tillader et hav af fortolkninger –- i kristendommens tilfælde dogmet om de dødes opstandelse. Dette dogme lærer, at der er en opstandelse, men overfor den efterfølgende evigheds hvordan og hvorledes, har kristendommen forholdt sig tøvende vedrørende en definitiv konklusion. 

Emnet om Guds nåde, og hvordan denne kommer til udtryk, synes at være punktet, hvor teologerne rammer en mur af guddommelig stilhed. Spørgsmålet er om denne guddommelige stilhed ikke forholder sig stille som udtryk for accept af menneskets ret til dets egne forståelse af livet og det, der følger efter døden?

Alle mennesket skal med sikkerhed dø på et tidspunkt, men hvad der kommer til at ske med os efter døden ved vi ikke. Derfor kommer vores tanker om døden utvivlsomt til at påvirke vores liv.