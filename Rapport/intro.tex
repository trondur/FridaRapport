\chapter{Introduktion}
Hvad bevæger vores liv sig hele tiden hen imod er et eksistentialistisk funderet spørgsmål. Kristendommen svar på dette spørgsmål er læren om de dødes opstandelse. Denne lære kommer under eskatologi, som betyder læren om de sidste tider og omhandler ethvert menneskes skæbne som død, dom, opstandelse, himmel og helvede. Indenfor kristendommen kan man finde tre teologiske hovedsynspunkter indenfor eskatologi: læren om dobbelt udgang, apokatastasislæren og annihilationsteorien. Disse tre synspunkter er forskellige opfattelser af menneskets sjælstilstand, da skiftet mellem liv og død indtræffer.

Som sagt omhandler eskatologi ethvert menneskes skæbne, men derudover omhandler den også menneskets håb på en kontinuitet efter dødens indtrædelse. Omdrejningspunktet for kristen eskatologi er Kristi genkomst, hvor han vil dømme levende og døde. Denne begivenhed er genstand for menneskets tro og håb på Guds frelse. I den forstand er dommen konstituerende for troen, idet den fungerer som et spejl og holder mennesket medansvarligt i livet.1 Spørgsmålet, hvad der forestilles ved dommen, har op igennem tiderne ændret karakter. De forskellige traditioner at tænke om dommen varierer i forhold til, hvordan Gud vælger at møde mennesket i sin nåde. Opgaven fremstiller Augustins, Regin Prenters, Gregor af Nyssas, Jürgen Moltmanns og Karl Barths (skildret af Niels Henrik Gregersens) forestillinger om den yderste dom. Med baggrund i deres forestillinger vil opgaven søge at opstille, hvad der er muligt at sige om de sidste tider. Endvidere vil der i opgaven, med ovenfornævnte forestillinger som baggrund, være en refleksion over, hvordan menneskets tanker om de sidste tider påvirker forståelsen af livet.