\chapter{Diskussion}
Gregersens artikel Karl Barth og annihilationsteorien indledes således: ”netop i evighedens lys skal der aflægges regnskab for hvert øjeblik. Netop over for evigheden bliver tiden dyrebar.”1 Ydermere er evigheden konstituerende og limitativt begreb for tiden. Som konstituerende for tiden betyder evigheden, at tiden til evigt forbliver en ’uindhentelig fremtid’ og i kristen forstand kommer mennesket derved til enhver tid at stå overfor Guds dom.2 Samtidig er evigheden et limitativt begreb, fordi åbenbarelsen af dommens dag forbliver i den evige fremtid.3 I den forstand står mennesket imellem en Guds dom og spørgsmålet om der overhoved er en Gud eller en dom. Mennesket er fanget midt imellem, fordi døden tillader intet andet. Døden fordrer mennesket til forestillinger om, hvad døden har som følge. Disse forestillinger forholder sig til spørgsmålet om Gud, nåde eller frelse. I forhold til teologerne Augustin, Prenter, Moltmann, Gregor, Barth og Gregersen forekommer sammenhængen i ens forestilling om efterlivet at afhænge af menneskets egen forståelse af livet her. Forstået på den måde, at livet bliver konstituerende for efterlivet. Dommen er nødvendig for den røde tråd, der forbinder livet og dets kontinuitet i dødens opstandelse.

Ifølge Gregor forestilling forekommer Guds vilje til frelse at være en selvfølgelighed. Dertil forudsætter slægtsforholdene mellem mennesket og Gud, at mennesket nødvendigvis drages i retning af Gud. Gregor har en tillid til selvfølgeligheden i Guds grænseløse kærlighed, der resulterer i alles frelse. Denne tiltro til Guds grænseløse kærlighed tillægger Moltmann også sin lære om det universale håb på frelse. Hans forestillingen forholder sig til en genoprettes der omfatter alle. Baggrunden herfor er håbet på en universel retfærdighed. I en refleksion over uretten livet ofte møder mennesket med, fremsætter Moltmann et håb på, at døden er en rehabiliteringsproces for mennesket. Det transcendente håb kommer for Moltmann at være forudsætningen for menneskets fulde oplevelse af dets eksistens. Det vil sige, at opstandelses håb sejrer over døden midt i livet og efterlader ikke mennesket i en cliff-hanger, som Moltmann forklarer således: ”I shall live wholly here, and die wholly, and rise wholly there.”4 Ifølge Moltmann er det ikke menneskets tro, der afgør dets frelse, men derimod skaber frelsen troen.5 Dette udtryk kan også gøre sig gældende for Gregors opfattelse.

På den anden side fordre døden, at man som menneske tænker over ens livsførelse. Sætninger som ”Lev som om du skulle dø i morgen”, ”Du ved aldrig, hvornår du har levet din sidste dag” og ”Memento mori” er udtryk for just dette. Tilsvarende disse udsagn synes Augustin og Prenter at være enige om, at livsførelse har noget at sige for, hvilket efterliv en idømmes. Augustin lægger vægt på, at ret livsførelse der søges igennem tro er kriteriet for belønning af evigt liv i himmelen. Ifølge Prenter forholder spørgsmålet om himmelen sig til en fornyelse der har retning mod fuldendelsen i herlighedens evighed. Fornyelsen ifølge Prenter er en livsførelse i fællesskab med kirkens, da kirkens menighed er Guds nye pagts gudsfolk. Grundlaget for ens frelse opfattes af Prenter at være forholdet til Kristus. Derimod forekommer det for dem, at et menneske, der ikke forholder sig til Kristus og Gud i sin livsførelse kommer at idømmes helvede som straf herfor.

Barths forestilling af menneskets livsmuligheder foreholder sig tilsvarende Augustin og Prenter til en dobbelt udgang. Modsat Augustin og Prenter foregår der ikke en opdeling af mennesker, men en opdeling i mennesket. Dommen går direkte ned igennem mennesket, hvor synden i mennesket skal tilintetgøres, men mennesket som synder skal reddes. I den forstand repræsenterer Barth sammen med Gregor og Moltmann frelsesuniversalisme.

Teologernes anstrengelser for at skabe en tese vedrørende Gud forbliver aldrig mere end just det, en tese. Så hvorfor gøre sig disse anstrengelser? En anden mulighed ville være at forholde sig til dette liv som om intet kom bagefter. Epikur mente, at ”mens vi er, er døden ikke. Mens døden er, er vi ikke.” Det vil sige, at der ingen transcendent verden findes udenfor vor erfaringskategorier. Mennesket oplever nu og engang holder det op med at opleve. Det er dog ikke således mennesket indretter sit verdens billede. Mennesket søger konstant at skabe mening ud af fortiden for at forholde sig til nutiden og fremtiden. Det vil sige, at mennesket i sine tanker krydser dødens grænse både baglens og forlæns. Ud af dette kan man sige, at forudsætningen for domforestillingerne er døden og det, at mennesket ikke kan acceptere den som blot en død. Måske giver forestillingerne livet lidt mere nuance og rumlighed.  