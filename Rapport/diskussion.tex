\chapter{Diskussion}
\label{chp:disk}
Gregersens artikel \textit{Karl Barth og annihilationsteorien} indledes således: "Netop i evighedens lys skal der aflægges regnskab for hvert øjeblik. Netop over for evigheden bliver tiden dyrebar."1 I en anden artikel \textit{Ind i Gud: Om at tænke opstandelse og evigt liv i dag} skriver Gregersen, at menneskets har en sekventiel tænkning. Det vil sige, at mennesket erkender stykkevis, med små bidder af gangen, og først i det evige liv kommer mennesket at forstå sammenhængen af helheden.\footnote{Gregersen, 2015, s. 1} Selv om den store sammenhæng unddrager vores forståelse, forsøger vi at sætte mening til tingene. Således giver vi tingene den mening, vores verdensbriller forsyner os med. Evigheden er et limitativt begreb, men samtidig er evigheden konstituerende for vores opfattelse af tiden som en evig fornyende effekt på tiden.\footnote{Gregersen, 2000, s. 1} Denne opfattelse af tiden muliggør tanken om evigheden som nærværende i tiden.\footnote{Gregersen, 2000, s. 1} På den måde er mennesket fanget midt imellem tiden og evigheden, og det udfordrer mennesket til at forestille sig, hvad døden har som følge. I forhold til teologerne Augustin, Prenter, Moltmann, Gregor og Barth forekommer der i deres forestillinger om efterlivet at være en sammenhæng mellem menneskets egen forståelse af livet og fortolkningen af efterlivet. Forstået på den måde, at livet her bliver konstituerende for fortolkningen af efterlivet. Dommen er det nødvendige bindeled mellem livet og dets kontinuitet i dødens opstandelse.

\section{Frelsesuniversalisme}
Ifølge Gregor forestilling er Guds vilje til alles frelse en selvfølgelighed. Dertil forudsætter slægtsforholdet mellem mennesket og Gud, at mennesket nødvendigvis drages i retning af Gud. Gregor udviser en enorm tillid til selvfølgeligheden i Guds grænseløse kærlighed, der medfører, at alle stort set tvinges til frelse. Denne grænseløse kærlighed er ligeledes at finde i Moltmann forestilling om alles frelse, som er baseret et universelt håb på frelse. Man kan spørge, om der overhovedet er tale om kærlighed, når Guds kærlighed tvinger menneskerne til at blive frelst? Der er snarere tale om ejerskab, hvor menneskets frie vilje ikke bliver respekteret, men hellere undermineret. Denne underminering af menneskets frie vilje er også det universalisme er under kritik for. I denne sammenhæng forsvarer Talbott tanken om universalisme idet en, der var fuldt informeret og indforstået med den sande Guds kærlighed og visdom, aldrig ville vælge andet end frelsen. Talbott påpeger en indre modsigelse i forståelsen af Gud inden for tanken om helvede. Modsigelsen omfatter, at Gud siges at ville alles frelse, hvorfor så formode, at han ikke i dette henseende kan gå sig selv i møde.\footnote{Talbott, s. 449} Derved konkluder Talbott med et udsagn af Jesus i John. 8:32: “og i skal lære sandheden at kende, og sandheden skal gøre jer frie.” Tilsvarende dette udsagn skriver Moltmann, at det ikke er menneskets tro, der afgør dets frelse, men derimod skaber frelsen troen.5 Gregors tanke om selvfølgeligheden i Guds kærlighed kommer til udtryk her idet, at Guds frelse ikke er baseret på det gode vi har gjort. Hans nåde er nåde for nådens skyld. Frelsesuniversalismen som livsopfattelse redder mennesket fra oplevelsen af, at stå over for en cliff-hanger ved døden. I stedet kan mennesket i denne opfattelse hvile i håbet, der er omfattes i frelsen just på grund af Guds kærlighed.

\section{Dobbelt udgang}
I opfattelsen af læren om dobbelt udgang er menneskeheden opdelt i to grupper, de frelste og de fortabte. Denne lære lægger vægt på menneskets livsførelse. Dette medfører, at mennesket er medansvarligt i forhold til dets skæbne. Næsten ingen har i sit fået at vide, at de er ansvarligt for deres egne beslutninger og målsætninger. Men der er stadig få, der ikke har erkendt, at det er sådan verden fungerer. Vi forventer simpelthen, at man tager ansvar for sig selv. Tilsvarende synes det at forholde sig for Augustin og Prenter, at livsførelse har noget at sige for, hvilket efterliv en idømmes. Som Augustin skriver i \textit{De civ. Dei 19,4}, så søges ret livsførelse igennem tro og dette er kriteriet for belønning af evigt liv i himmelen. Det vil sige, at mennesket, ifølge denne forestilling af dom, fortjener evig straf, hvis ikke det tager ansvaret for sin egen skæbne og holder sig til Gud igennem dets livsførelse. Problemet med denne forestilling er, at mennesket ikke altid er klar over konsekvenserne af en handling, selv om den blev gjort med bedste hensigt. Dog forholder dommen ifølge Prenter sig til, om man tror på Kristus eller ej. Men er det ikke et for stort ansvar at lægge på mennesket, at tro eller vantro, forstået som forseelse mod Gud, skal være afgørende for ens evige skæbne. Det er blandt andet i dette forhold, læren om dobbelt udgang har modtaget kritik. For er det en nådig og kærlig Gud, der tillader mennesket, at skabe helvedes straf for sig selv? Kvanvig gør i denne forbindelse opmærksom på, at det, at en forseelse mod Gud retfærdiggør menneskets straf i helvede, er et skrøbeligt argument. For mennesket har generelt ikke intentioner om at forse sig mod Gud, når de gør det. Hvis Guds nåde skal forholde sig til mennesket på denne måde, så er den ikke længere guddommelig og uovertruffen, men hevet ned på menneskeligt niveau. Derimod forholder universalismen sig til mennesket som menneske, da den accepterer at mennesker kan fejle. I modsætning til hvad Augustin mener, at ondskaben er et redskab for Gud til at frelse de gode, så tilbyder frelsesuniversalismen virkelig en Gud, der forstår at skabe det gode ud af det dårlige.

\section{Annihilation}
Barths forestilling om menneskets livsmuligheder forholder sig tilsvarende Augustins og Prenters forestillinger om en dobbelt udgang. Modsat hvad Augustin og Prenter mener, så mener Barth ikke at der foregår en opdeling af mennesker, men en opdeling midt i ethvert menneske, således at dommen går direkte ned igennem mennesket og tilintetgør synden og frelser mennesket. I denne sammenhæng repræsenterer Barth en form for frelsesuniversalisme. I hans forstand er Guds nåde uafhængig af menneskets foranderlige følelsesliv. Det vil sige, at Guds "Ja" til mennesket ikke er afhængig af menneskets "Ja" til Gud.\footnote {Gregersen, 2000, s. 136-139} Tanken om Gud som uovertruffen i vilje og magt nødvendiggør logisk, at han ikke er afhængig af mennesket, der halvdelen af tiden ikke har magt over sine egne tanker, følelser og handlinger. Tilsvarende dette gør Pinnock opmærksom på, at der i Guds straf må være en sammenhæng i således, at finit synd ikke kan straffes med infinit straf. Dog holdes mennesket medansvarligt, fordi dommen er stadig. Det er bare sådan, at Gud i Barths optik ikke tillader mennesket at komme så meget til skade, at det aldrig kan vende tilbage fra skaden. 

\section{Human- eller glashusforestilling}
Forestillingerne med frelsesuniversalistiske motiver forekommer at være mere humane idet, der i disse udvises en respekt for, at alle mennesker kan fejle. Dette forstået som, at det ikke er fejlene eller manglen på det samme, der afgør et menneskes værdi. Disse forestillinger fremstiller en tanke om, at alle mennesker er ligeværdige til Guds frelse. Derimod forekommer læren om dobbelt udgang at lægge mennesket i lænker. Mennesket skal, ud fra disse opfattelser af Guds nåde, hele tiden være bange for sig selv. I forhold til en slig opfattelse af frelsen og fortabelsen, kommer mig i hug ordsproget: Man ikke skal kaste med sten, når man selv bor i et glashus. Det er undringsværdig, hvilke slags mennesker det er, der tillægger sig og prædiker for en slig domsforestilling.